\section{Consultas simples}

\begin{frame}[fragile]{Consultas simples}

    \begin{itemize}
        \item Uma vez que a base de dados do interpretador Prolog seja alimentado com fatos, o
            este interpretador pode responder a consultas (\textit{queries}) a respeito dos
            fatos

        \item As consultas em Prolog funcionam por meio do casamento de padrões
            (\textit{pattern matching})

        \item O padrão de uma consulta é denominado \textbf{objetivo} (\textit{goal})

        \item Se algum fato atinge o objetivo, a consulta é bem sucedida e o interpretador
            responde ``\texttt{Sim}'' (\code{prolog}{true.})

        \item Caso contrário, a consulta falha e o interpretador responde ``\texttt{Não}''
            (\code{prolog}{false.})

        \item O casamento de padrões do Prolog é denominado \textbf{unificação}

    \end{itemize}

\end{frame}

\begin{frame}[fragile]{Unificação}

    \begin{block}{Unificação (versão simplificada)}
        Se o programa contém apenas fatos, a unificação é bem sucedida se as três condições
            abaixo são satisfeitas:
        \begin{enumerate}
            \item o predicado citado no objetivo e na base de dados é o mesmo,
            \item ambos tem a mesma aridade,
            \item todos os argumentos são os mesmos.
        \end{enumerate}

    \end{block}

\end{frame}

\begin{frame}[fragile]{Exemplos de unificação}

    \inputsnippet{prolog}{1}{21}{codes/matching.pl}

\end{frame}

\section{Estruturas de seleção}

\begin{frame}[fragile]{Estruturas de seleção}

    \begin{itemize}
        \item Smalltalk permite a escrita de blocos de código que só serão executados caso uma
            condição seja verdadeira (ou falsa)

        \item Isto é feito de acordo paradigma da orientação aos objetos: a condição gera um
            objeto do tipo booleano, o qual recebe uma mensagem cujo argumento é um bloco, que
            também é um objeto

        \item De fato, um bloco é um objeto composto por \textit{statements} executáveis

        \item Assim, a estrutura de seleção  \texttt{if-else}, comum em outras linguagens, de 
            fato não é uma forma da linguagem Smalltalk, e sim um comportamento dos objetos
            booleanos

    \end{itemize}

\end{frame}

\begin{frame}[fragile]{Métodos \texttt{ifTrue} e \texttt{ifFalse}}

    \begin{itemize}
        \item O método \code{smalltalk}{ifTrue} recebe como argumento um bloco de comandos e o
            executa, caso o objeto seja \code{smalltalk}{true}, ou retorna sem executá-lo, caso
            o objeto seja \code{smalltalk}{false}

            \inputsyntax{smalltalk}{codes/iftrue.st}

        \item O método \code{smalltalk}{ifFalse} tem comportamento análogo, executando o bloco
            caso o objeto seja \code{smalltalk}{false} e retornando sem executar se o objeto é
            \code{smalltalk}{true}

            \inputsyntax{smalltalk}{codes/iffalse.st}

        \item Eles podem ser usados para simular o comportamento do \texttt{if-else} padrão das
            linguagens imperativas:

            \inputsyntax{smalltalk}{codes/ifelse.st}
    \end{itemize}

\end{frame}

\begin{frame}[fragile]{Exemplo de uso de condicionais em SmallTalk}
    \inputsnippet{smalltalk}{1}{20}{codes/quadratic.st}
\end{frame}

\begin{frame}[fragile]{Exemplo de uso de condicionais em SmallTalk}
    \inputsnippet{smalltalk}{21}{41}{codes/quadratic.st}
\end{frame}

\section{Programação Procedural}

\begin{frame}[fragile]{Programação Procedural}

    \begin{itemize}
        \item A ideia que originou a programação procedural surgiu por volta de 1958, antes do 
            paradigma estruturado estar completamente estabelecido

        \item O objetivo era diminuir a complexidade dos programas, dividindo-os em unidades
            menores

        \item Cada unidade, denominada \textbf{procedimento}, era responsável por uma única tarefa

        \item Estes procedimentos seriam equivalentes aos verbos nas linguagens naturais

        \item Este paradigma se desenvolveu como uma evolução do paradigma estruturado, sendo 
            comum usar ambos termos combinados (programação estruturada/procedural) ou mesmo
            como sinônimos

    \end{itemize}

\end{frame}

\begin{frame}[fragile]{Exemplo de linguagem procedural: C}

    \begin{itemize}
        \item A linguagem C foi desenvolvida em 1972 por Ken Thompson e Dennis Ritchie

        \item Ela combina construtos de alto nível com elementos de baixo nível (ponteiros,
            \code{c}{goto}, etc)

        \item A partir do padrão estabelecido em 1988, os códigos escritos em C se tornaram
            portáveis para todas as plataformas que tivessem um compilador C

        \item C é uma linguagem adequada para programação de sistemas operacionais, compiladores,
            jogos e aplicações comerciais, e tem sido amplamente utilizada desde sua criação

        \item Por exemplo, na linguagem C, o programa é representado pela função 

            \inputsyntax{c}{codes/main.st}
    \end{itemize}

\end{frame}

\begin{frame}[fragile]{Exemplo de linguagem procedural: C}

    \begin{itemize}
        \item O retorno da função \code{c}{main()} é capturado pelo sistema operacional

        \item Zero significa que o programa finalizou sua execução com sucesso; qualquer outro
            valor representa um possível erro na execução

        \item Os dois parâmetros (os quais podem ser omitidos) representam o número de parâmetros
            (\code{c}{argc}) e os parâmetros (\code{c}{argv}) passados em linha de comando

        \item As subrotinas são declaradas sem retorno:

            \inputsyntax{c}{codes/sub.st}

        \item As funções e subrotinas podem invocar outras funções e subrotinas, ou mesmo a si
            próprias

        \item Uma função/subrotina que invoca a si mesma é chamada \textbf{recursiva}
    \end{itemize}

\end{frame}

\begin{frame}[fragile]{Exemplo de linguagem procedural: C}

    \begin{itemize}
        \item Cada função/subrotina deve ser definida/implementada em um único ponto, mas pode
            ser declarada ou invocada quantas vezes forem necessárias

        \item A sintaxe para a declaração de uma função é

            \inputsyntax{c}{codes/decl.st}

        \item Para a definição da função, a sintaxe é
            \inputsyntax{c}{codes/f.st}

        \item O tipo do retorno e dos parâmetros é um dos tipos primitivos (\code{c}{char},
            \code{c}{int}, \code{c}{float} e \code{c}{double}), ou ponteiros para estes tipos
            (ou para o tipo \code{c}{void}), ou tipos definidos pelo usuário (\code{c}{struct} ou
                \code{c}{union})

        \item O valor a ser retornado deve seguir o comando \code{c}{return}

        \item Para referenciar uma função implementada em outro arquivo, a declaração deve ser
            antecedida pela palavra reservada \code{c}{extern}
    \end{itemize}

\end{frame}

\begin{frame}[fragile]{Exemplo de código procedural escrito em C}
    \inputsnippet{c}{1}{22}{codes/main.c}
\end{frame}

\begin{frame}[fragile]{Exemplo de código procedural escrito em C}
    \inputsnippet{c}{1}{22}{codes/coin_change.h}
\end{frame}

\begin{frame}[fragile]{Exemplo de código procedural escrito em C}
    \inputsnippet{c}{1}{22}{codes/coin_change.c}
\end{frame}

\begin{frame}[fragile]{Exemplo de código procedural escrito em C}
    \inputsnippet{c}{23}{45}{codes/coin_change.c}
\end{frame}

\begin{frame}[fragile]{Exemplo de código procedural escrito em C}
    \inputsnippet{c}{46}{68}{codes/coin_change.c}
\end{frame}

\begin{frame}[fragile]{Exemplo de código procedural escrito em C}
    \inputsnippet{c}{69}{81}{codes/coin_change.c}
\end{frame}

\section{Subrotinas e Funções}

\begin{frame}[fragile]{Subrotinas e Funções}

    \begin{itemize}
        \item Em Fortran, uma \textbf{subrotina} difere de uma \textbf{função} no
            sentido de que não possui um valor de retorno

        \item Ambas podem ser declaradas no próprio arquivo do programa, ou em arquivos
            separados

        \item Funções são invocadas da mesma maneira que as funções intrínsecas da linguagem

        \item As subrotinas são invocadas por meio de um comando \code{fortran}{call}

        \item A comunicação entre o programa e as funções e subrotinas se dá por meio de 
            \textbf{argumentos} (ou \textbf{parâmetros}) e do \textbf{retorno}, no caso
            das funções

        \item Ambas são fundamentais em programas estruturados, no sentido que permite a
            organização e reuso de trechos de código, formando unidades semânticas
    \end{itemize}

\end{frame}

\begin{frame}[fragile]{Funções}

    \begin{itemize}
        \item A sintaxe para a declaração de uma função é a seguinte:

            \inputsyntax{fortran}{codes/function.st}

        \item O retorno da função deve armazenado em uma variável local de mesmo nome da função

        \item O bloco de comandos pode ser encerrado prematuramente, por meio do comando
            \code{fortran}{return}

        \item As variáveis são passadas por \code{fortran}{referência}

        \item A primeira implicação deste fato é que os parâmetros devem ter o mesmo tipo da
            variável passada como parâmetro na chamada

        \item A segunda implicação é que, caso um parâmetro seja modificado na função, esta
            mudança será feita na variável original

        \item As funções devem ser declaradas a partir do ponto marcado pela palavra-chave
            \code{fortran}{contains}
    \end{itemize}

\end{frame}

\begin{frame}[fragile]{Exemplo de declaração e uso de funções}
    \inputsnippet{fortran}{1}{22}{codes/binom.f90}
\end{frame}

\begin{frame}[fragile]{Exemplo de declaração e uso de funções}
    \inputsnippet{fortran}{23}{45}{codes/binom.f90}
\end{frame}

\begin{frame}[fragile]{Subrotinas}

    \begin{itemize}
        \item A sintaxe para a declaração de subrotinas é semelhante à declaração de funções:

            \inputsyntax{fortran}{codes/subroutines.st}

        \item Assim como as funções, as subrotinas recebem os valores de seus argumentos por
            referência, o que permite a modificação destes parâmetros

        \item Não há retorno em subrotinas

        \item As subrotinas também devem ser declaradas após a palavra-chave 
            \code{fortran}{contains}, e encerradas a qualquer momento por meio do comando
            \code{fortran}{return}

    \end{itemize}

\end{frame}

\begin{frame}[fragile]{Exemplo de declaração e uso de subrotinas}
    \inputsnippet{fortran}{1}{22}{codes/selection.f90}
\end{frame}

\begin{frame}[fragile]{Exemplo de declaração e uso de subrotinas}
    \inputsnippet{fortran}{23}{42}{codes/selection.f90}
\end{frame}

\begin{frame}[fragile]{Exemplo de declaração e uso de subrotinas}
    \inputsnippet{fortran}{43}{66}{codes/selection.f90}
\end{frame}

\begin{frame}[fragile]{Observações sobre funções e subrotinas}

    \begin{itemize}
        \item No caso em que um dos parâmetros é um vetor \code{fortran}{xs} de tamanho 
            desconhecido, as dimensões deste pode ser obtido por meio da função intrínseca
            \code{fortran}{size()}

        \item Para tal, na declaração do tipo de parâmetro esta dimensão desconhecida deve ser
            indicada (por exemplo, \code{fortran}{integer :: xs(:)})

        \item Como um subrotina pode usar um parâmetro tanto para entrada como para saída, o uso
            de cada parâmetro pode ser explicitado por meio do atributo \code{fortran}{intent}

        \item O parâmetro atributo é um dentre três valores possíveis: \code{fortran}{in},
            \code{fortran}{out} e \code{fortran}{inout}

        \item Além de melhorar a legibilidade, este atributo previne que um parâmetro de entrada
            seja modificado

        \item O atributo \code{fortran}{save} pode ser utilizado para marcar variáveis locais que
            mantém seus valores entre as chamadas de uma função ou subrotina 
    \end{itemize}

\end{frame}

\begin{frame}[fragile]{Exemplo de subrotina em Fortran}
    \inputsnippet{fortran}{1}{22}{codes/fibonacci.f90}
\end{frame}

\begin{frame}[fragile]{Exemplo de subrotina em Fortran}
    \inputsnippet{fortran}{23}{45}{codes/fibonacci.f90}
\end{frame}

\section{Variáveis}

\begin{frame}[fragile]{Variáveis em Fortran}

    \begin{itemize}
        \item Em Fortran, as variáveis simbolizam regiões de memória, as quais podem ser 
            lidas ou escritas

        \item Cada variável é identificada por um \textbf{nome}, que deve iniciar com um
            caractere alfabético e conter apenas caracteres alfanuméricos ou o símbolo 
            \verb|'_'|

        \item Em Fortran não há distinção entre caracteres maiúsculos e minúsculos

        \item Assim como nas linguagens imperativas, uma variável identifica tanto o endereço
            da região de memória quanto o valor armazenado

        \item Qual dos dois valores será utilizado depende do contexto (se é um \textit{l-value} ou
            um \textit{r-value}

    \end{itemize}

\end{frame}

\begin{frame}[fragile]{Declaração de variáveis e tipos de dados}

    \begin{itemize}
        \item Uma variável pode ser declarada em Fortran usando a seguinte syntaxe

            \inputsyntax{fortran}{codes/variables.st}

        \item Os principais tipos de dados em Fortran são: \code{fortran}{real}, 
            \code{fortran}{integer}, \code{fortran}{complex} e \code{fortran}{character}
        
        \item O valor inicial é opcional

        \item Strings podem ser declaradas indicando-se o número de caracteres que a compõe

            \inputsyntax{fortran}{codes/strings.st}

        \item Para declarar \textbf{constantes}, isto é, variáveis com permissão para leitura
            apenas), é utilizada a palavra-chave \code{fortran}{parameter}:

            \inputsyntax{fortran}{codes/parameter.st}
    
        \item No caso de constantes, o valor inicial é mandatório

        \item A expressão \code{fortran}{implicit none} determina que todas as variáveis devem
            ser declaradas antes de seu uso, e é boa prática sempre utilizá-la no início dos
            programas
    \end{itemize}

\end{frame}

\begin{frame}[fragile]{Exemplo de declaração e uso de variáveis em Fortran}
    \inputsnippet{fortran}{1}{22}{codes/area.f90}
\end{frame}

\begin{frame}[fragile]{Operadores aritméticos e funções intrínsecas}

    \begin{itemize}
        \item Sendo uma linguagem voltada para computação científica, Fortran tem suporte para
            uma série de operadores aritméticos

        \item No caso das expressões com mais de um operador, o operador de menor precedência é
            computado antes do de menor precedência
        \item Além disso, há um bom número de funções \textit{intrísecas} da linguagem, disponíveis
            sem a necessidade de importar arquivos ou bibliotecas externas

        \item Boa parte destas funções são relacionadas às funções matemáticas e manipulação
            numérica
    \end{itemize}

    \begin{table}[ht]
        \centering
        \begin{tabular}{ccl}
            \toprule
            \textbf{Operador} & \textbf{Precedência} & \textbf{Operação} \\
            \midrule
                \texttt{**} & 1 & Expoenciação \\
                \texttt{*} & 2 & Multiplicação \\
                \texttt{/} & 2 & Divisão \\
                \texttt{+} & 3 & Adição \\
                \texttt{-} & 3 & Subtração \\
            \bottomrule
        \end{tabular}
    \end{table}
\end{frame}

\begin{frame}[fragile]{Funções intrísecas úteis}

    \begin{table}[ht]
        \centering
        \begin{tabular}{ll}
            \toprule
            \textbf{Função} & \textbf{Retorno} \\
            \midrule
                \texttt{abs(a)} & Valor absoluto de \texttt{a} \\
                \texttt{sin(w)} & Seno de \texttt{w} \\
                \texttt{cos(w)} & Cosseno de \texttt{w} \\
                \texttt{tan(x)} & Tangente de \texttt{w} \\
                \texttt{sqrt(x)} & Raiz quadrada de \texttt{x} \\
                \texttt{conjg(z)} & Conjugado complexo de \texttt{z} \\
                \texttt{log10(x)} & Logaritmo em base 10 de \texttt{x} \\
                \texttt{mod(r1, r2)} & Resto da divisão de \texttt{r1} por \texttt{r2} \\
                \texttt{max(r1, r2, ...)} & Maior dentre todos os argumentos \\
                \texttt{min(r1, r2, ...)} & Menor dentre todos os argumentos \\
            \bottomrule
        \end{tabular}
    \end{table}

    \vspace{0.2in}

    \textbf{Legenda}: {\texttt{r}: real ou inteiro, \texttt{z}: complexo, \texttt{w}: real ou 
        complexo, \texttt{x}: real, \texttt{a}: qualquer tipo.}
\end{frame}

\begin{frame}[fragile]{Exemplo de uso de funções intrísecas e operadores aritméticos}
    \inputsnippet{fortran}{1}{22}{codes/polar.f90}
\end{frame}

\begin{frame}[fragile]{Operadores lógicos e relacionais}

    \begin{itemize}
        \item Fortran também tem suporte para variáveis booleanas, cujos valores possíveis 
            são verdadeiro (\code{fortran}{.TRUE.}) e falso (\code{fortran}{.FALSE.})

        \item As variáveis booleanas são declaradas com o tipo \code{fortran}{logical}:

            \inputsyntax{fortran}{codes/logical.st}

        \item Variáveis boolenas ou expressões que resultem em valores booleanos podem ser 
            combinadas com os operadores lógicos \textbf{e} (\code{fortran}{.and.}), \textbf{ou}
            (\code{fortran}{.or}) ou \textbf{não} (\code{fortran}{.not})

        \item Os operadores relacionais são apresentados em duas formas

        \item A primeira delas é a em notação símbolica: \code{fortran}{<, <=, ==, >=, >, /=}

        \item A segunda é por meio de com operadores semelhantes aos operadores lógicos:
            \code{fortran}{.lt., .le., .eq., .ge., .gt., .ne.}
    \end{itemize}

\end{frame}

\begin{frame}[fragile]{Exemplo de uso de operadores relacionais}
    \inputsnippet{fortran}{1}{22}{codes/maratona.f90}
\end{frame}

\section{Manipulação da pilha}

\begin{frame}[fragile]{Palavras para a manipulação da pilha}

    \begin{itemize}
        \item Além da palavra \code{forth}{.}, que extrai o topo da pilha, Forth disponibiliza
            outras palavras para a manipulação da pilha

        \item A palavra \code{forth}{SWAP ( a b -- b a )} inverte a ordem do topo com o segundo
            elemento da pilha:

            \inputcode{forth}{codes/swap.fs}
            \vspace{0.1in}

        \item A palavra \code{forth}{DUP ( a -- a a )} duplica o elemento do topo da pilha:

            \inputcode{forth}{codes/dup.fs}

    \end{itemize}

\end{frame}

\begin{frame}[fragile]{Palavras para a manipulação da pilha}

    \begin{itemize}
        \item A palavra \code{forth}{OVER ( a b -- a b a )} insere uma copia o segundo elemento na pilha:

            \inputcode{forth}{codes/over.fs}
            \vspace{0.1in}

        \item A palavra \code{forth}{ROT ( a b c -- b c a )} rotaciona o terceiro elemento, de modo que
            ele passa a ocupar o topo da pilha:

            \inputcode{forth}{codes/rot.fs}
            \vspace{0.1in}

        \item A palavra \code{forth}{DROP ( a -- )} remove o topo da pilha, sem imprimí-lo na saída
            padrão

            \inputcode{forth}{codes/drop.fs}

    \end{itemize}

\end{frame}

\begin{frame}[fragile]{Sumário}

    \begin{scriptsize}
    \begin{table}
        \centering
        \begin{tabularx}{0.95\textwidth}{ccX}
            \toprule
            \textbf{Palavra} & \textbf{Notação de pilha} & \textbf{Significado} \\
            \midrule
            \code{forth}{SWAP} & \code{forth}{( a b -- b a )} & Troca os dois elementos do topo da pilha de posição \\
            \midrule
            \code{forth}{DUP} & \code{forth}{( a -- a a )} & Duplica o elemento do topo da pilha \\
            \midrule
            \code{forth}{SPACES} & \code{forth}{( n -- )} & Imprime $n$ espaços \\
            \midrule
            \code{forth}{EMIT} & \code{forth}{( c -- )} & Imprime o caractere $c$ \\
            \midrule
            \code{forth}{."} & \code{forth}{( -- )} & Imprime a string delimitada por \code{forth}{"} que se segue\\
            \midrule
            \code{forth}{.} & \code{forth}{( n -- )} & Imprime o número $n$, seguido de um espaço $c$ \\
            \midrule
            \code{forth}{+} & \code{forth}{( n1 n2 -- s )} & Computa $s = n1 + n2$ \\
            \midrule
            \code{forth}{-} & \code{forth}{( n1 n2 -- s )} & Computa $s = n1 - n2$ \\
            \midrule
            \code{forth}{*} & \code{forth}{( n1 n2 -- m )} & Computa $m = n1 \times n2$ \\
            \midrule
            \code{forth}{/} & \code{forth}{( n1 n2 -- q )} & Computa o quociente $q$ da divisão inteira de $n1$ por $n2$ \\
            \midrule
            \code{forth}{MOD} & \code{forth}{( n1 n2 -- r )} & Computa o resto $r$ da divisão inteira de $n1$ por $n2$ \\
            \midrule
            \code{forth}{/MOD} & \code{forth}{( n1 n2 -- q r )} & Computa o quociente $q$ e o resto $r$ da divisão de $n1$ por $n2$ \\
            \bottomrule
        \end{tabularx}
    \end{table}
    \end{scriptsize}

\end{frame}



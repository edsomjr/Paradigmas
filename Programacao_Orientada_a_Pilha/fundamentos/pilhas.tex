\section{Programação Funcional}

\begin{frame}[fragile]{Visão geral}

    \begin{itemize}
        \item A programação funcional é um paradigma de programação que trata a computação como
            a valoração de funções matemáticas e que evita estados e dados mutáveis

        \item Tem como fundamento o cálculo lambda

        \item Os programas consistem em expressões, de modo que é um paradigma declarativo, não
            imperativo

        \item O conceito principal é de que o retorno de uma função depende única e exclusivamente
            de seus parâmetros (ou argumentos)

        \item Isto significa que uma função chamada duas ou mais vezes, com os mesmos
            argumentos, 
            terá sempre o mesmo retorno, o que não necessariamente ocorre em paradigmas
            imperativos

    \end{itemize}

\end{frame}

\begin{frame}[fragile]{Exemplo em C de mesmos argumentos, retornos distintos}
    \inputcode{c}{codes/function.c}
\end{frame}

\begin{frame}[fragile]{Histórico da programação funcional}

    \begin{itemize}
        \item A programação funcional tem raízes no cálculo lambda, que surgiu na década de 1930

        \item Na década de 1950, Lisp foi uma das primeiras linguagens a incorporar conceitos e
            características da programação funcional

        \item Na década de 1960 a APL (\textit{A Programming Language}) foi desenvolvida e 
            influenciou o desenvolvimento da programação funcional na década seguinte

        \item Também na décade de 1970 foi desenvolvida a linguagem ML e em 1987 teve início o
            desenvolvimento da linguagem Haskell

        \item Na década de 1990 surgiram as linguagens J, K e Q

        \item Com o passar do tempo e da maturação destas linguagens, elas deixaram de ser usadas
            apenas em pesquisas e na academia e hoje são utilizadas em aplicações reais
    \end{itemize}

\end{frame}

\begin{frame}[fragile]{Principais características da programação funcional}

    \begin{enumerate}[(a)]
        \item As funções são cidadãs de primeira classe: as funções podem aceitar funções
            como parâmetros e podem retornar funções

        \item As funções são puras, isto é, não há efeitos colaterais

        \item Uso extensivo de recursão

        \item As variáveis são imutáveis

        \item A valoração das expressões é não-estrita, de modo que o valor de uma variável ou
            função só é computado quando for utilizado (também chamada \textit{lazy evaluation})

        \item As expressões dos programas tem um valor de retorno 

        \item Uso de \textit{pattern matching} para extração dos membros de objetos ou para
            condicionais
    \end{enumerate}

\end{frame}

\begin{frame}[fragile]{Variáveis imutáveis}

    \begin{itemize}
        \item Variáveis \textbf{imutáveis} são aquelas cujo valor não pode ser modificado após sua
            inicialização

        \item Em linguagens imperativas, tais variáveis são denominadas \textbf{constantes}

        \item Variáveis imutáveis são úteis em ambientes \textit{multithread}, pois elas não levam
            a problemas de concorrência

        \item Em linguagens funcionais, uma variável é atada a um valor ou expressão em sua
            atribuição inicial, e ao longo do programa ela não pode ser desatada ou atada a uma
            outra expressão ou valor

        \item O código Haskell abaixo gera um erro de compilação, pois a variável
            \texttt{x} é imutável:

            \inputsyntax{haskell}{codes/immutable.hs}
    \end{itemize}
\end{frame}

\begin{frame}[fragile]{Funções de primeira classe}

    \begin{itemize}
        \item \textbf{Funções de primeira classe} são funções que recebem o mesmo tratamento que
            os tipos primitivos da linguagem

        \item Assim, elas podem ser parâmetros ou retornos de outras funções, ou podem ser
            armazenadas em variáveis

        \item \textbf{Funções de alta ordem} são funções que recebem uma ou mais funções como
            parâmetros, ou que retornam uma função

        \item Funções que não são de alta ordem são denominadas funções de \textbf{primeira ordem}

        \item No cálculo lambda, todas as funções de são alta ordem

        \item Um exemplo típico de função de alta ordem é a função \code{haskell}{map()}, que 
            recebe como parâmetros uma função \code{haskell}{f()} e uma lista \code{haskell}{xs},
            e e retorna uma lista cujo $i$-ésimo elemento é $f(x_i)$

        \item Exemplo em Haskell:
            \inputsyntax{haskell}{codes/map.hs}
    \end{itemize}

\end{frame}

\begin{frame}[fragile]{Funções anônimas}

    \begin{itemize}
        \item As funções contém quatro partes: nome, lista de parâmetros, corpo e retorno

        \item \textbf{Funções anônimas} são funções que não possuem um nome

        \item Há dois tipos de funções anônimas: \textbf{funções lambda} e 
            \textit{\textbf{closures}}

        \item Em geral, funções anônimas são utilizadas como parâmetros de funções de alta ordem,
            ou para a construção dos retornos destas

        \item Funções lambda são funções compostas por uma lista de parâmetros, um corpo e um
            retorno

        \item \textit{Closures} são semelhantes às funções lambda, exceto pelo fato de que elas
            referenciam variáveis foram do escopo do seu corpo e de sua lista de parâmetros

        \item Em geral, os \textit{closures} são implementados como estruturas de dados que
            contém um ponteiro para o código da função e para todas as variáveis necessárias para
            sua execução
    \end{itemize}

\end{frame}

\begin{frame}[fragile]{Funções puras e efeitos colaterais}

    \begin{itemize}
        \item Uma função ou expressão tem um \textbf{efeito colateral} se, além de computar o
            valor de retorno, ela modifica o estado do programa

        \item Os efeitos colaterais mais comuns são: alteração de variáveis globais, estáticas ou
            parâmetros, escrita ou leitura em periféricos, e invocação de funções que tenham
            efeitos colaterais

        \item A ausência de efeitos colaterais é uma condição necessária, porém não suficiente,
            para a transparência referencial

        \item A \textbf{transparência referencial} significa que uma expressão pode ser substituída
            por seu valor

        \item Funções \textbf{puras} não tem efeitos colaterais e geram o mesmo retorno para o mesmo conjunto de parâmetros

        \item Em outras palavras, o retorno depende única e exclusivamente dos parâmetros
            
    \end{itemize}

\end{frame}

\begin{frame}[fragile]{\it Currying}

    \begin{itemize}
        \item \textit{Currying} é uma técnica de transformação que converte uma função com
            múltiplos parâmetros em uma sequência de funções de um único parâmetro cada, cujos
            retornos são as próximas funções da sequência

        \item Por exemplo, considere a função 
        \begin{align*}
            f: \mathbb{Z} \times \mathbb{Z} & \to \mathbb{Z} \\
               (x, y) & \mapsto f(x, y) = x + y
        \end{align*}

        \item A aplicação do \textit{currying} em $f$ leva as funções
        \begin{align*}
            g: \mathbb{Z} & \to (\mathbb{Z} \to \mathbb{Z}) \\
               x & \mapsto g(x) = h_x(y)
        \end{align*}
        e
        \begin{align*}
            h_x: \mathbb{Z} & \to \mathbb{Z} \\
               y & \mapsto h_x(y) = x + y
        \end{align*}
    \end{itemize}

\end{frame}

\begin{frame}[fragile]{Valoração não-estrita}

    \begin{itemize}
        \item No contexto de implementação de linguagens de programação, valoração é o 
            processo de computar o valor de uma expressão

        \item Valoração \textbf{estrita} significa que as expressões são valoradas imediatamente e
            atribuídas à variável que conterá o retorno tão logo esta última seja definida

        \item Em chamadas de função, a valoração estrita computa o valor de todos os parâmetros
            antes da execução do bloco da função

        \item Já a valoração \textbf{não-estrita} não computa o valor de um variável, mesmo em
            sua definição, postergando este cálculo para o seu primeiro uso

        \item A valoração não-estrita também é conhecida como \textit{lazy evaluation}

        \item Por exemplo, o código abaixo não gera um laço infinito, mesmo sendo a lista
            infinita, pois ela só é computada até o valor especificado

            \inputsyntax{haskell}{codes/take.hs}
    \end{itemize}

\end{frame}

\begin{frame}[fragile]{\it Pattern matching}

    \begin{itemize}
        \item Em programação funcional, \textit{pattern matching} diz respeito ao processo de
            verificação de objeto contra objeto

        \item Ela é utilizada para extrair membros dos objetos ou verificar se os objetos tem
            um tipo específico em uma única expressão

        \item Isto leva a menos linhas de código com atribuições de variáveis e uma melhor
            leitura e entendimento do código

        \item Em geral, são escritas vários padrões possíveis para o objeto, e o objeto será
            confrontado com cada um deles, na sequência apresentada, até que se encontre
            um padrão que corresponda ao objeto

        \item É comum o uso de um caractere ou palavra-chave que corresponda a um padrão
            universal, que corresponderá a qualquer objeto (\textit{wildcard})

            \inputsyntax{haskell}{codes/factorial.hs}

    \end{itemize}

\end{frame}

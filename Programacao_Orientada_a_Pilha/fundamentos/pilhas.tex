\section{Programação Orientada a Pilhas}

\begin{frame}[fragile]{Visão geral}

    \begin{itemize}
        \item A programação orientada a pilhas é um paradigma de programação que utiliza uma ou
            mais pilhas para a manipulação de dados e passagem de parâmetros

        \item O modelo de computação é o das máquinas de Turing

        \item A maioria das linguagens que suportam esse paradigma usam a notação pós-fixada para
            suas operações, isto é, os argumentos são informados antes da operação

        \item Como as operações manipulam a pilha, adicionando e removendo dados, é comum manter
            um registro das modificações feitas na pilha, denominado \textbf{diagrama de efeitos na
            pilha} (\textit{stack effects diagram})

        \item Expressões e programas podem ser interpretados de maneira simples e direta

        \item Exemplos de linguagens orientadas a pilhas: Forth, PostScript, Bibtex e Uiua

    \end{itemize}

\end{frame}

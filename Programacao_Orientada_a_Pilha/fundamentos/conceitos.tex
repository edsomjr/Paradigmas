\section{Conceitos elementares de Forth}

\begin{frame}[fragile]{Palavras}

    \begin{itemize}
        \item Forth organiza seus códigos por meio de comandos nomeados, que abstram tarefas
            ou ações correlacionadas por meio de um nome comum

        \item Estes comandos nomeados seriam os equivalentes a funções em outras linguagem

        \item A sintaxe para a criação de um novo comando é

            \inputsyntax{forth}{codes/word.fs}
        
        ou seja, inicia com dois-pontos, segue com o nome e continua com a definição, que termina
        com ponto-e-vírgula

        \item Comandos definidos dessa forma são denominados \textbf{palavras} (\textit{words})

        \item O padrão ANS da linguagem disponibiliza um conjunto grande (mais de 200) palavras
            pré-definidas, que podem ser usadas para definir novas palavras

        \item A habilidade de definir palavras por meio de palavras já definidas é denominada
            \textbf{extensibilidade}
    \end{itemize}

\end{frame}

\section{Conceitos elementares de Forth}

\begin{frame}[fragile]{Palavras}

    \begin{itemize}
        \item Forth organiza seus códigos por meio de comandos nomeados, que abstram tarefas
            ou ações correlacionadas por meio de um nome comum

        \item Estes comandos nomeados seriam os equivalentes a funções em outras linguagem

        \item A sintaxe para a criação de um novo comando é

            \inputsyntax{forth}{codes/word.fs}
        
        ou seja, inicia com dois-pontos, segue com o nome e continua com a definição, que termina
        com ponto-e-vírgula

        \item Comandos definidos dessa forma são denominados \textbf{palavras} (\textit{words})

        \item O padrão ANS da linguagem disponibiliza um conjunto grande (mais de 300) palavras
            pré-definidas, que podem ser usadas para definir novas palavras

        \item A habilidade de definir palavras por meio de palavras já definidas é denominada
            \textbf{extensibilidade}
    \end{itemize}

\end{frame}


\begin{frame}[fragile]{Modo interativo}

    \begin{itemize}
        \item No modo interativo do Forth (REPL), o interpretador responde aos comandos executados
            de forma bem sucedida com a palavra ``\code{forth}{ok}''

        \item Por exemplo, entre no modo interativo e digite a tecla \texttt{ENTER}: Forth responderá 
            ``\code{forth}{ok}'', movendo o cursor para a próxima linha

        \item Para inserir um ou mais números na pilha, basta inseri-los, na ordem desejada e 
            separados por um espaço em branco, e digitar \texttt{ENTER}

        \item Para visualizar o estado atual da pilha, use a palavra \code{forth}{.s}, a qual não
            tem parâmetros

            \inputcode{forth}{codes/stack.fs}
    \end{itemize}
\end{frame}

\begin{frame}[fragile]{Palavras para a manipulação do terminal}

    \begin{itemize}
        \item Quando uma palavra recebe um ou mais argumentos, eles são extraídos da pilha, 
            do último para o primeiro

        \item Por exemplo, a palavra \code{forth}{spaces} recebe um argumento $n$ e imprime $n$
            espaços no terminal

            \inputcode{forth}{codes/spaces.fs}

        \vspace{0.1in}

        \item A palavra \code{forth}{emit} recebe um inteiro $n$ imprime no terminal o caractere
            cujo código ASCII é $n$:

            \inputcode{forth}{codes/emit.fs}

    \end{itemize}

\end{frame}

\begin{frame}[fragile]{Palavras para a manipulação do terminal}

    \begin{itemize}
        \item O código abaixo define uma nova palavra, chamada \code{forth}{STAR}, que imprime um 
            asterisco no terminal:

            \inputcode{forth}{codes/star.fs}

        \vspace{0.1in}

        \item Também é possível definir uma nova palavra, chamada \code{forth}{STARTS}, que recebe
            um argumento $n$ e imprime $n$ asteriscos consecutivos:

            \inputcode{forth}{codes/stars.fs}

    \end{itemize}

\end{frame}

\begin{frame}[fragile]{O dicionário}

    \begin{itemize}
        \item Todas as palavras definidas em Forth, seja em biblioteca padrão, seja pelo usuário, são
            armazenadas no ``dicionário''

        \item Quando uma nova palavra é definida, Forth compila a palavra e a insere em seu dicionário

        \item Por exemplo, a linha abaixo é uma definição alternativa para a palavra \code{forth}{STAR}:
            (\code{forth}{[char]} traduz o caractere para seu código ASCII):

            \inputsyntax{forth}{codes/star2.fs}

        \item Quando um comando é inserido no terminal, 
            será ativada a palavra \code{forth}{interpret}, que fará a leitura da entrada em busca
            de uma string (sequência de caracteres separada por espaços em branco)
    \end{itemize}

\end{frame}

\begin{frame}[fragile]{O dicionário}

    \begin{itemize}
        \item Se a palavra encontrada consta no dicionário,
            será ativada a palavra \code{forth}{execute}, que executa a definição da palavra e 
            finaliza com a mensagem ``\code{forth}{ok}''

        \item Se a palavra não consta no dicionário, então Forth tentará interpretar a string
            como um número: caso ele tenha sucesso na conversão, o número lido será 
            inserido na pilha

        \item Se a palavra lida não é um número, Forth sinaliza um erro, indicando que a palavra
            não foi definida

        \item Os nomes das novas palavras devem ser compostos por, no máximo, 31 caracteres imprimíveis
    \end{itemize}

\end{frame}

\section{Forth}

\begin{frame}[fragile]{Forth}

    \begin{itemize}
        \item Forth é uma linguagem baseada em pilhas, desenvolvida por Charles H. Moore nos anos 70

        \item Segundo o próprio autor, ela foi desenvolvida enquanto ele trabalhava em um IBM 1130, 
            um computador de terceira geração

        \item Ele considerou que os resultados que tinha até momento eram tão bons que ele 
            considerou que estava desenvolvendo uma ``\textit{fourth-generation computer language}''

        \item Daí surgiu a ideia de nomeá-la FOURTH

        \item Porém o IBM 1130 permita identificadores com, no máximo, cinco caracteres

        \item Por este motivo ele deu à linguagem o nome FORTH

        \item Forth foi utilizada pela NASA no desenvolvimento de aplicações para missões espaciais

    \end{itemize}
\end{frame}

\begin{frame}[fragile]{GForth}

    \begin{itemize}
        \item GForth é uma implementação \textit{open source} da linguagem

        \item Em Linux, ele pode ser instalado com o comando

            \inputsyntax{bash}{codes/install.sh}

        \item Para checar se a instalação foi bem sucedida, rode o comando

            \inputsyntax{haskell}{codes/forth.sh}

        \item O GForth tem dois modos de operação: o interativo (REPL) e o modo interpretador,
            onde ele lê e executa as instruções contidas em um \textit{script} Forth

        \item Para entrar no modo REPL, basta invocar o interpretador com o comando abaixo

            \inputsyntax{haskell}{codes/repl.sh}

        \item Para encerrar a sessão, utilize a palavra \code{forth}{bye}
    \end{itemize}

\end{frame}

\begin{frame}[fragile]{Interpretador GForth}

    \begin{itemize}

        \item Para utilizar o GForth no modo interpretador, deve se escrever os comandos Forth
            em um arquivo de texto (\textit{script})

        \item A extensão adotada para \textit{scripts} Forth é \texttt{.fs}

        \item Para interpretar um \textit{script} Forth basta invocar o GForth, passando como 
            argumento o nome do \textit{script}

            \inputsyntax{bash}{codes/run.sh}

        \item O \textit{script} abaixo implementa o tradicional ``\textit{Hello World!}'' em Forth:

            \inputcode{forth}{codes/hello.fs}
    \end{itemize}

\end{frame}

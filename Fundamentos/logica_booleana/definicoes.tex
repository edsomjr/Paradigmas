\section{Conceitos elementares}

\begin{frame}[fragile]{Lógica Proposicional Booleana}

    \begin{block}{Termos Primitivos}
        Os termos primitivos da Lógica Proposicional Booleana são: 
        \begin{enumerate}
            \item proposição
            \item verdadeiro
            \item falso
        \end{enumerate}
    \end{block}

    \begin{block}{Axiomas}
       \begin{description}
            \item[Princípio do Terceiro Excluído:] uma proposição é verdadeira ou é falsa.
            \item[Princípio da Não-Contradição:] uma proposição não pode ser, simultaneamente, verdadeira e falsa.
        \end{description}
    \end{block}

\end{frame}

\begin{frame}[fragile]{Exemplos}

    Exemplos de proposições:

    \begin{itemize}
        \item A duração de um dia é de 24 horas
        \item A metade de dois mais dois é igual a três
        \item $F_n = 2^{2^n} + 1$ é primo para qualquer $n$ natural
    \end{itemize}

    \vspace{0.3in}

    Não são proposições:

    \begin{itemize}
        \item Cuidado! Curva acentuada à esquerda!
        \item Onde fica a agência bancária mais próxima?
        \item $x > \pi$
    \end{itemize}

\end{frame}

\begin{frame}[fragile]{Proposições Compostas}

    \begin{block}{Operadores Lógicos}
        Sejam $p$ e $q$ duas proposições. São proposições:
        \begin{enumerate}
            \item a \textbf{conjunção} $p \land q$: verdadeira somente quando ambas $p$ e $q$ são verdadeiras
            \item a \textbf{disjunção} $p \lor q$: falsa somente quando ambas $p$ e $q$ são falsas
            \item a \textbf{disjunção exclusiva} $p \veebar q$: falsa somente quando ambas $p$ e $q$ tem mesmo valor lógico
            \item a \textbf{condicional} $p \to q$: falsa somente quando $p$ é verdadeira e $q$ é falsa
            \item a \textbf{bicondicional} $p \leftrightarrow q$: falsa somente quando $p$ e $q$ tem valores lógicos distintos
            \item a \textbf{negação} $\lnot p$: verdadeira quando $p$ é falsa, falsa quando $p$ é verdadeira
        \end{enumerate}
    \end{block}

\end{frame}

\begin{frame}[fragile]{Exemplos de proposições compostas}

    \begin{itemize}
        \item ``O meu pai era paulista / Meu avô, pernambucano / O meu bisavô, mineiro / Meu tataravô, baiano / Meu maestro soberano
            / foi Antonio Brasileiro'' (Paratodos, Chico Buarque)
        \item ``Ser ou não ser.'' (Hamlet, William Shakespeare)
        \item ``Penso, logo existo.'' (René Descartes)
        \item ``Um conjunto de $\mathbb{R}^{n}$ é sequencialmente compacto se, e somente se, é fechado e limitado.'' (Teorema de Bolzano-Weierstrass) 
        \item ``Não pode ser seu amigo quem exige seu silêncio.'' (Alice Walker)
    \end{itemize}

\end{frame}

\begin{frame}[fragile]{Tabela-Verdade}
    \begin{block}{Tabela-Verdade}
        Uma \textbf{tabela-verdade} é uma representação visual na qual figuram todos os possíveis
        valores lógicos de uma proposição composta correspondentes a todas as possíveis atribuições
        de valores lógicos às proposições simples componentes.
    \end{block}

    \vspace{0.2in}

    \begin{block}{Proposição}
        Seja $P(q_1, q_2, \ldots, q_N)$ uma proposição composta. Então a tabela verdade de $P$ contém
        $2^N$ linhas.
    \end{block}

\end{frame}

\begin{frame}[fragile]{Exemplo: tabela-verdade de $P = p \land q \to r$}

    \begin{table}
        \centering
        \begin{tabular}{>{\tt}c>{\tt}c>{\tt}c>{\tt}c>{\tt}c}
            \hline
            $p$ & $q$ & $r$ & \textcolor{white}{$p\land q$} & \textcolor{white}{$P$}\\
            \hline
            \textcolor{white}{V} & \textcolor{white}{V} & \textcolor{white}{V} & \textcolor{white}{V} & \textcolor{white}{V} \\
            \hline
            \textcolor{white}{V} & \textcolor{white}{V} & \textcolor{white}{V} & \textcolor{white}{V} & \textcolor{white}{V} \\
            \hline
            \textcolor{white}{V} & \textcolor{white}{V} & \textcolor{white}{V} & \textcolor{white}{V} & \textcolor{white}{V} \\
            \hline
            \textcolor{white}{V} & \textcolor{white}{V} & \textcolor{white}{V} & \textcolor{white}{V} & \textcolor{white}{V} \\
            \hline
            \textcolor{white}{V} & \textcolor{white}{V} & \textcolor{white}{V} & \textcolor{white}{V} & \textcolor{white}{V} \\
            \hline
            \textcolor{white}{V} & \textcolor{white}{V} & \textcolor{white}{V} & \textcolor{white}{V} & \textcolor{white}{V} \\
            \hline
            \textcolor{white}{V} & \textcolor{white}{V} & \textcolor{white}{V} & \textcolor{white}{V} & \textcolor{white}{V} \\
            \hline
            \textcolor{white}{V} & \textcolor{white}{V} & \textcolor{white}{V} & \textcolor{white}{V} & \textcolor{white}{V} \\
            \hline
        \end{tabular}
    \end{table}
\end{frame}

\begin{frame}[fragile]{Exemplo: tabela-verdade de $P = p \land q \to r$}

    \begin{table}
        \centering
        \begin{tabular}{>{\tt}c>{\tt}c>{\tt}c>{\tt}c>{\tt}c}
            \hline
            $p$ & $q$ & $r$ & \textcolor{white}{$p\land q$} & \textcolor{white}{$P$}\\
            \hline
            \textcolor{black}{V} & \textcolor{white}{V} & \textcolor{white}{V} & \textcolor{white}{V} & \textcolor{white}{V} \\
            \hline
            \textcolor{black}{V} & \textcolor{white}{V} & \textcolor{white}{V} & \textcolor{white}{V} & \textcolor{white}{V} \\
            \hline
            \textcolor{black}{V} & \textcolor{white}{V} & \textcolor{white}{V} & \textcolor{white}{V} & \textcolor{white}{V} \\
            \hline
            \textcolor{black}{V} & \textcolor{white}{V} & \textcolor{white}{V} & \textcolor{white}{V} & \textcolor{white}{V} \\
            \hline
            \textcolor{white}{V} & \textcolor{white}{V} & \textcolor{white}{V} & \textcolor{white}{V} & \textcolor{white}{V} \\
            \hline
            \textcolor{white}{V} & \textcolor{white}{V} & \textcolor{white}{V} & \textcolor{white}{V} & \textcolor{white}{V} \\
            \hline
            \textcolor{white}{V} & \textcolor{white}{V} & \textcolor{white}{V} & \textcolor{white}{V} & \textcolor{white}{V} \\
            \hline
            \textcolor{white}{V} & \textcolor{white}{V} & \textcolor{white}{V} & \textcolor{white}{V} & \textcolor{white}{V} \\
            \hline
        \end{tabular}
    \end{table}
\end{frame}

\begin{frame}[fragile]{Exemplo: tabela-verdade de $P = p \land q \to r$}

    \begin{table}
        \centering
        \begin{tabular}{>{\tt}c>{\tt}c>{\tt}c>{\tt}c>{\tt}c}
            \hline
            $p$ & $q$ & $r$ & \textcolor{white}{$p\land q$} & \textcolor{white}{$P$}\\
            \hline
            \textcolor{black}{V} & \textcolor{white}{V} & \textcolor{white}{V} & \textcolor{white}{V} & \textcolor{white}{V} \\
            \hline
            \textcolor{black}{V} & \textcolor{white}{V} & \textcolor{white}{V} & \textcolor{white}{V} & \textcolor{white}{V} \\
            \hline
            \textcolor{black}{V} & \textcolor{white}{V} & \textcolor{white}{V} & \textcolor{white}{V} & \textcolor{white}{V} \\
            \hline
            \textcolor{black}{V} & \textcolor{white}{V} & \textcolor{white}{V} & \textcolor{white}{V} & \textcolor{white}{V} \\
            \hline
            \textcolor{black}{F} & \textcolor{white}{V} & \textcolor{white}{V} & \textcolor{white}{V} & \textcolor{white}{V} \\
            \hline
            \textcolor{black}{F} & \textcolor{white}{V} & \textcolor{white}{V} & \textcolor{white}{V} & \textcolor{white}{V} \\
            \hline
            \textcolor{black}{F} & \textcolor{white}{V} & \textcolor{white}{V} & \textcolor{white}{V} & \textcolor{white}{V} \\
            \hline
            \textcolor{black}{F} & \textcolor{white}{V} & \textcolor{white}{V} & \textcolor{white}{V} & \textcolor{white}{V} \\
            \hline
        \end{tabular}
    \end{table}
\end{frame}

\begin{frame}[fragile]{Exemplo: tabela-verdade de $P = p \land q \to r$}

    \begin{table}
        \centering
        \begin{tabular}{>{\tt}c>{\tt}c>{\tt}c>{\tt}c>{\tt}c}
            \hline
            $p$ & $q$ & $r$ & \textcolor{white}{$p\land q$} & \textcolor{white}{$P$}\\
            \hline
            \textcolor{black}{V} & \textcolor{black}{V} & \textcolor{white}{V} & \textcolor{white}{V} & \textcolor{white}{V} \\
            \hline
            \textcolor{black}{V} & \textcolor{black}{V} & \textcolor{white}{V} & \textcolor{white}{V} & \textcolor{white}{V} \\
            \hline
            \textcolor{black}{V} & \textcolor{black}{F} & \textcolor{white}{V} & \textcolor{white}{V} & \textcolor{white}{V} \\
            \hline
            \textcolor{black}{V} & \textcolor{black}{F} & \textcolor{white}{V} & \textcolor{white}{V} & \textcolor{white}{V} \\
            \hline
            \textcolor{black}{F} & \textcolor{white}{V} & \textcolor{white}{V} & \textcolor{white}{V} & \textcolor{white}{V} \\
            \hline
            \textcolor{black}{F} & \textcolor{white}{V} & \textcolor{white}{V} & \textcolor{white}{V} & \textcolor{white}{V} \\
            \hline
            \textcolor{black}{F} & \textcolor{white}{V} & \textcolor{white}{V} & \textcolor{white}{V} & \textcolor{white}{V} \\
            \hline
            \textcolor{black}{F} & \textcolor{white}{V} & \textcolor{white}{V} & \textcolor{white}{V} & \textcolor{white}{V} \\
            \hline
        \end{tabular}
    \end{table}
\end{frame}


\begin{frame}[fragile]{Exemplo: tabela-verdade de $P = p \land q \to r$}

    \begin{table}
        \centering
        \begin{tabular}{>{\tt}c>{\tt}c>{\tt}c>{\tt}c>{\tt}c}
            \hline
            $p$ & $q$ & $r$ & \textcolor{white}{$p\land q$} & \textcolor{white}{$P$}\\
            \hline
            \textcolor{black}{V} & \textcolor{black}{V} & \textcolor{white}{V} & \textcolor{white}{V} & \textcolor{white}{V} \\
            \hline
            \textcolor{black}{V} & \textcolor{black}{V} & \textcolor{white}{V} & \textcolor{white}{V} & \textcolor{white}{V} \\
            \hline
            \textcolor{black}{V} & \textcolor{black}{F} & \textcolor{white}{V} & \textcolor{white}{V} & \textcolor{white}{V} \\
            \hline
            \textcolor{black}{V} & \textcolor{black}{F} & \textcolor{white}{V} & \textcolor{white}{V} & \textcolor{white}{V} \\
            \hline
            \textcolor{black}{F} & \textcolor{black}{V} & \textcolor{white}{V} & \textcolor{white}{V} & \textcolor{white}{V} \\
            \hline
            \textcolor{black}{F} & \textcolor{black}{V} & \textcolor{white}{V} & \textcolor{white}{V} & \textcolor{white}{V} \\
            \hline
            \textcolor{black}{F} & \textcolor{black}{F} & \textcolor{white}{V} & \textcolor{white}{V} & \textcolor{white}{V} \\
            \hline
            \textcolor{black}{F} & \textcolor{black}{F} & \textcolor{white}{V} & \textcolor{white}{V} & \textcolor{white}{V} \\
            \hline
        \end{tabular}
    \end{table}
\end{frame}

\begin{frame}[fragile]{Exemplo: tabela-verdade de $P = p \land q \to r$}

    \begin{table}
        \centering
        \begin{tabular}{>{\tt}c>{\tt}c>{\tt}c>{\tt}c>{\tt}c}
            \hline
            $p$ & $q$ & $r$ & \textcolor{white}{$p\land q$} & \textcolor{white}{$P$}\\
            \hline
            \textcolor{black}{V} & \textcolor{black}{V} & \textcolor{black}{V} & \textcolor{white}{V} & \textcolor{white}{V} \\
            \hline
            \textcolor{black}{V} & \textcolor{black}{V} & \textcolor{black}{F} & \textcolor{white}{V} & \textcolor{white}{V} \\
            \hline
            \textcolor{black}{V} & \textcolor{black}{F} & \textcolor{black}{V} & \textcolor{white}{V} & \textcolor{white}{V} \\
            \hline
            \textcolor{black}{V} & \textcolor{black}{F} & \textcolor{black}{F} & \textcolor{white}{V} & \textcolor{white}{V} \\
            \hline
            \textcolor{black}{F} & \textcolor{black}{V} & \textcolor{black}{V} & \textcolor{white}{V} & \textcolor{white}{V} \\
            \hline
            \textcolor{black}{F} & \textcolor{black}{V} & \textcolor{black}{F} & \textcolor{white}{V} & \textcolor{white}{V} \\
            \hline
            \textcolor{black}{F} & \textcolor{black}{F} & \textcolor{black}{V} & \textcolor{white}{V} & \textcolor{white}{V} \\
            \hline
            \textcolor{black}{F} & \textcolor{black}{F} & \textcolor{black}{F} & \textcolor{white}{V} & \textcolor{white}{V} \\
            \hline
        \end{tabular}
    \end{table}
\end{frame}

\begin{frame}[fragile]{Exemplo: tabela-verdade de $P = p \land q \to r$}

    \begin{table}
        \centering
        \begin{tabular}{>{\tt}c>{\tt}c>{\tt}c>{\tt}c>{\tt}c}
            \hline
            $p$ & $q$ & $r$ & \textcolor{black}{$p\land q$} & \textcolor{white}{$P$}\\
            \hline
            \textcolor{black}{V} & \textcolor{black}{V} & \textcolor{black}{V} & \textcolor{black}{V} & \textcolor{white}{V} \\
            \hline
            \textcolor{black}{V} & \textcolor{black}{V} & \textcolor{black}{F} & \textcolor{black}{V} & \textcolor{white}{V} \\
            \hline
            \textcolor{black}{V} & \textcolor{black}{F} & \textcolor{black}{V} & \textcolor{black}{F} & \textcolor{white}{V} \\
            \hline
            \textcolor{black}{V} & \textcolor{black}{F} & \textcolor{black}{F} & \textcolor{black}{F} & \textcolor{white}{V} \\
            \hline
            \textcolor{black}{F} & \textcolor{black}{V} & \textcolor{black}{V} & \textcolor{black}{F} & \textcolor{white}{V} \\
            \hline
            \textcolor{black}{F} & \textcolor{black}{V} & \textcolor{black}{F} & \textcolor{black}{F} & \textcolor{white}{V} \\
            \hline
            \textcolor{black}{F} & \textcolor{black}{F} & \textcolor{black}{V} & \textcolor{black}{F} & \textcolor{white}{V} \\
            \hline
            \textcolor{black}{F} & \textcolor{black}{F} & \textcolor{black}{F} & \textcolor{black}{F} & \textcolor{white}{V} \\
            \hline
        \end{tabular}
    \end{table}
\end{frame}

\begin{frame}[fragile]{Exemplo: tabela-verdade de $P = p \land q \to r$}

    \begin{table}
        \centering
        \begin{tabular}{>{\tt}c>{\tt}c>{\tt}c>{\tt}c>{\tt}c}
            \hline
            $p$ & $q$ & $r$ & \textcolor{black}{$p\land q$} & \textcolor{black}{$P$}\\
            \hline
            \textcolor{black}{V} & \textcolor{black}{V} & \textcolor{black}{V} & \textcolor{black}{V} & \textcolor{black}{V} \\
            \hline
            \textcolor{black}{V} & \textcolor{black}{V} & \textcolor{black}{F} & \textcolor{black}{V} & \textcolor{black}{F} \\
            \hline
            \textcolor{black}{V} & \textcolor{black}{F} & \textcolor{black}{V} & \textcolor{black}{F} & \textcolor{black}{V} \\
            \hline
            \textcolor{black}{V} & \textcolor{black}{F} & \textcolor{black}{F} & \textcolor{black}{F} & \textcolor{black}{V} \\
            \hline
            \textcolor{black}{F} & \textcolor{black}{V} & \textcolor{black}{V} & \textcolor{black}{F} & \textcolor{black}{V} \\
            \hline
            \textcolor{black}{F} & \textcolor{black}{V} & \textcolor{black}{F} & \textcolor{black}{F} & \textcolor{black}{V} \\
            \hline
            \textcolor{black}{F} & \textcolor{black}{F} & \textcolor{black}{V} & \textcolor{black}{F} & \textcolor{black}{V} \\
            \hline
            \textcolor{black}{F} & \textcolor{black}{F} & \textcolor{black}{F} & \textcolor{black}{F} & \textcolor{black}{V} \\
            \hline
        \end{tabular}
    \end{table}
\end{frame}



\begin{frame}[fragile]{Sentença aberta}

    \begin{block}{Sentença aberta (informal)}
        Uma sentença aberta $S(x)$ em $x$ é uma expressão na qual o símbolo $x$ ocorre uma 
        ou mais vezes e que, caso todas as ocorrências de $x$ sejam substituídas por um 
        mesmo valor $v$, $S(v)$ se torna uma proposição.
    \end{block}

\end{frame}

\begin{frame}[fragile]{Quantificadores}

    \begin{block}{Quantificador existencial}
        Seja $S(x)$ uma sentença aberta. O quantificador existencial $\exists$ é utilizado na construção $\exists x.S(x)$, a qual
        significa que existe pelo menos um $x$ tal que $S(x)$ é verdadeira.
    \end{block}

    \vspace{0.3in}

    \begin{block}{Quantificador universal}
        Seja $S(x)$ uma sentença aberta. O quantificador universal $\forall$ é utilizado na construção $\forall x.S(x)$, a qual
        significa que, para todos os valores de $x$, $S(x)$ é verdadeira.
    \end{block}
\end{frame}

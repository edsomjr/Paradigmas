\section{Computabilidade e Tese de Turing}

\begin{frame}[fragile]{Especificação para uma função de $k$ argumentos}

    \begin{enumerate}[(a)]
        \item Os argumentos $m_1, m_2, \ldots, m_k$ são apresentados em notação monádica por
            $k$ blocos com $m_i$ traços cada; os blocos são separados por um único espaço em branco
            e a fita, de resto, está em branco
        \item O computador começa examinando o traço mais à esquerda do bloco $m_1$; \textbf{(a)}
            e \textbf{(b)} caracterizam a \textbf{configuração inicial} da máquina
        \item Se $f(m_1, m_2, \ldots, m_k) = n$, a máquina para no traço mais à esquerda de um
            bloco contendo $n$ traços; de resto, a fita está em branco. Esta é a 
            \textbf{configuração (posição) final padrão}
        \item Se a função $f$ não está definida para os argumentos dados, ou a máquina não irá
            parar, ou irá parar em uma configuração final que não é a padrão
    \end{enumerate}

\end{frame}

\begin{frame}[fragile]{Exemplo de máquina que segue a especificação}

    \begin{itemize}
        \item Considere a máquina
        \[
            q_111q_2,
        \]
        que representa uma função de um único argumento $m$

        \item Ela examina o primeiro traço do bloco de $m$ traços, escreve $1$ (o que equivale
            a não fazer nada) e segue para o estado $2$

        \item A máquina para no estado $2$: neste momento, ela está sobre o quadrado mais à
            esquerda de um bloco de $m$ traços; de resto, a fita está vazia

        \item Logo a máquina para na configuração final padrão, e $f(m) = m$ para todo inteiro
            positivo $m$

        \item Assim, $f(x) = \mathrm{id}(x)$
    \end{itemize}

\end{frame}

\begin{frame}[fragile]{Computabilidade por Máquina de Turing}

    \begin{block}{Definição}
        Uma função numérica de $k$ argumentos é \textbf{computável por Máquina de Turing}, ou
        \textbf{Turing computável} se existe uma máquina que atenda as especificações 
        apresentadas e que compute $f(x)$ para todos os elementos $x$ no domínio de $f$.
    \end{block}

\end{frame}

\begin{frame}[fragile]{A Tese de Turing}

    \begin{block}{Tese de Turing}
        Toda função efetivamente computável é Turing computável.
    \end{block}

    \vspace{0.2in}

    \textbf{Observações}: naturalmente, toda função Turing computável é efetivamente computável.
    Note também que, uma vez que a noção de computabilidade não é rigorosamente definida, não é
    possível demonstrar formalmente a Tese de Turing, de modo que ela é, de fato, uma conjectura.
\end{frame}

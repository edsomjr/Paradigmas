\section{Entrada e saída em console}

\begin{frame}[fragile]{Código puro e código impuro}

    \begin{itemize}
        \item Haskell determina uma clara separação entre código puro e código impuro

        \item Esta estratégia permite que código puro fique isento de efeitos colaterais

        \item Além de facilitar a divisão semântica do código, ela permite aos compiladores
            otimizar e paralelizar trechos de código automaticamente

        \item Como as rotinas de entrada e saída interagem com o mundo externo, todas elas
            produzem ou estão suscetíveis a efeitos colaterais, sendo assim, códigos impuros
    \end{itemize}

\end{frame}

\begin{frame}[fragile]{Leitura e escrita de strings em console}

    \begin{itemize}
        \item Haskell provê um conjunto de funções para escrita e leitura de dados a partir 
            do console

        \item No que diz respeito à strings, duas funções básicas são \code{haskell}{putStrLn}
            e \code{smalltalk}{getLine}

        \item A função \code{haskell}{putStrLn} escreve uma string no console, seguida de uma
            quebra de linha, e tem tipo

            \inputsyntax{haskell}{codes/putstrln.hs}

        \item Já a função \code{haskell}{getLine} lê uma string do console até encontrar uma 
            quebra de linha e a retorna, sem a quebra

        \item O tipo da função \code{haskell}{getLine} é

            \inputsyntax{haskell}{codes/getline.hs}
    \end{itemize}

\end{frame}

\begin{frame}[fragile]{Exemplo de leitura e escrita de strings em console}
    \inputcode{haskell}{codes/echo.hs}
\end{frame}

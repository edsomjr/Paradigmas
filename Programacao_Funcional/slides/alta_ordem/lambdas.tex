\section{Funções anônimas}

\begin{frame}[fragile]{Funções lambda}

    \begin{itemize}
        \item Haskell permite a definição de funções anônimas, denominadas \textbf{funções lambda}

        \item A sintaxe para a definição de uma função lambda é

            \inputsyntax{haskell}{codes/lambda.hs}

        \item A lista de parâmetros \texttt{var1, var2, ..., varN} pode conter casamentos de
            padrões

        \item A expressão, contudo, não pode conter guardas

        \item A depender do contexto, pode ser necessário usar parêntesis para delimitar o
            corpo da função lambda

        \item Por exemplo, a função abaixo imprime os inteiros de 1 a $n$, substituindo os 
            múltiplos de $m$ pela palavra \code{haskell}{"Pim"}

            \inputsyntax{haskell}{codes/pim.hs}
    \end{itemize}

\end{frame}

\begin{frame}[fragile]{\it Currying}

    \begin{itemize}
        \item A aplicação parcial de uma função (\textit{currying}) é uma técnica de programação
            funcional que permite obter uma nova função através da aplicação incompleta de
            seus parâmetros
    \end{itemize}

\end{frame}

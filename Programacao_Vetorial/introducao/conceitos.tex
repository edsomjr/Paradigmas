\section{Conceitos elementares da APL}

\begin{frame}[fragile]{Características da APL}

    \begin{itemize}
        \item APL é uma linguagem dinamicamente tipada, interpretada e interativa

        \item Foi fortemente influenciada pela notação matemática

        \item Também  influenciou as linguagens J, K, Mathematica, MATLAB, Nial, PPL, Q; as planilhas eletrônicas, a programação funcional e pacotes matemáticos computacionais

        \item Ela suporta predominantemente a programação vetorial, de modo que a estrutra de dados básica para armazenamento de dados é o \textit{array}

        \item APL foca na solução do problema, enfatizando a expressão de algoritmos que independem da arquitetura ou do sistema operacional

        \item Ela promove o \textit{problem solving} em um maior nível de abstração

        \item A generalidade de APL provém da simplicidade e uniformidade de suas regras
    \end{itemize}

\end{frame}

\begin{frame}[fragile]{Características da APL}

    \begin{itemize}
        \item APL automatiza os aspectos irrelevantes da programação, ampliando a produtividade

        \item Outro aspecto importante é que APL trata os números e suas conversões internamente, tornando desnecessário o conhecimento de suas representações, limites, etc

        \item Dada a natureza da APL, alguns a consideram uma linguagem ``\textit{write-only}''

        \item À primeira vista a leitura de um código APL remete à decodficação de hieroglifos egípcios

        \item A concisão e abrangência das funções de APL permitem escrever o Jogo da Vida de Conway em uma única linha!

            \inputsyntax{apl}{codes/life.apl} 
    \end{itemize}

\end{frame}

\begin{frame}[fragile]{Glifos}

    \begin{itemize}
        \item Códigos APL utilizam glifos (símbolos), alguns oriundos do alfabeto grego, ao invés dos caracteres ASCII tradicionalmente usados em outras linguagens

        \item Alguns destes glifos são oriundos da própria matemática. Por exemplo: \texttt{+, -, ×, ÷, !, <, ≤, =, ≥, >, ≠, ∨, ∧, ∊}

        \item Linguagens convencionais substituem alguns destes símbolos por outros símbolos ASCII


        \item Por exemplo, C/C++ utiliza. \texttt{*} para multiplicação e  \texttt{/} para divisão

        \item APL denomina \textbf{primitivas} as características intrínsecas da linguagem, as quais são representadas por um ou mais símbolos (glifos)

        \item A maioria das primitivas são funções e operadores
    \end{itemize}

\end{frame}

\begin{frame}[fragile]{Ordem de avaliação das expressões}

    \begin{itemize}
        \item Em APL há apenas uma única e simples regra de precedência: o argumento à direita de uma função é o resultado de toda expressão à sua direita

        \item Esta ordem de avaliação parece estranha à princípio, mas traz vantagens

        \item Na ordem matemática, o algoritmo de Horner para computar o valor do polinômio $p(x) = a + bx + cx^2 + dx^3$ em $y$ seria notado como
        \[
            p(y) = a + y \times (b + y \times (c + y \times d))
        \]


            \inputsyntax{apl}{codes/horner.apl}

        \item Outro exemplo, em APL \texttt{2 × 3 + 5} resulta em 16, e não 11

        \item Esta escolha reduz substancialmente a necessidade de parêntesis
    \end{itemize}

\end{frame}

\begin{frame}[fragile]{Funções, \textit{arrays} e operadores}

    \begin{itemize}
        \item APL distingue entre funções e operadores

        \item As funções recebem \textit{arrays} retangulares como argumentos e retornam \textit{arrays}

        \item \textit{Arrays} retangulares tem zero ou mais dimensões, não necessariamente de mesmo tamanho

        \item Um vetor é um \textit{array} de dimensão 1, uma matriz um \textit{array} de dimensão 2

        \item Operadores permitem construir funções que são variantes de outras funções

        \item Eles remetem à funções de alta ordem e recebe funções ou \textit{arrays} como argumentos e derivam funções relacionadas

        \item Por exemplo, a função \texttt{sum} pode ser derivada a partir o operador \texttt{/} (redução) e da função \texttt{+} (adição)

            \inputsyntax{apl}{codes/sum.apl}
    \end{itemize}

\end{frame}

\begin{frame}[fragile]{Programas em APL}

    \begin{itemize}
        \item Em uma sessão interativa, um ambiente APL é denominado um \textbf{workspace}, onde o usuário pode inserir e manipular dados sem definir um programa

        \item Programas em APL são cadeias de funções monádicas ou diádicas em conjunto com operadores e \textit{arrays}

        \item Exemplo de programa APL, que ilustra a expressividade e concisão de APL: a função \texttt{P} abaixo retorna 1 se o argumento é um palíndromo, 0 caso contrário

            \inputsyntax{apl}{codes/p.apl}

        \item \textbf{Significado}: é verdade que, para todos os caracteres (\code{apl}{∧/}) do argumento \code{apl}{⍵}, eles são iguais (\code{apl}{=}) ao caractere correspondente do reverso (\code{apl}{⌽⍵}) deste argumento?
    \end{itemize}

\end{frame}

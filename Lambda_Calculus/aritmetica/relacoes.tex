\section{Relações entre números naturais}

\begin{frame}[fragile]{Teste Condicional}

    \begin{block}{Função $\mathbf{Z}$}
        O termo-$\lambda$
        \[
            Z\equiv \lambda x.xF\lnot F
        \]
        o qual chamaremos \textbf{função} $\mathbf{Z}$, retorna verdadeiro ($T$) quando aplicada
        em $0$, e retorna falso ($F$) para qualquer outro número natural.
    \end{block}

\end{frame}

\begin{frame}[fragile]{Observações sobre a função $\mathbf{Z}$}

    \begin{itemize}
        \item Para entender o comportamento da função $\mathbf{Z}$, observe que
        \[
            0yx \equiv (\lambda sz.z)yx \equiv x,
        \]
        isto é, quando aplicada ao termo $xy$, $0$ ignora o termo $y$ e retorna o termo
        $x$

        \item O termo-$\lambda$ $F$, quando aplicado em qualquer termo lambda $z$, 
        retorna a identidade $\mathbf{I}$, pois
        \[
            Fz \equiv (\lambda xy.y)z \equiv (\lambda x.(\lambda y.y))z \equiv (\lambda y.y)[x:=z]
                \equiv \lambda y.y \equiv \mathbf{I}
        \]

        \item O natural $N$ aplica $N$ vezes o termo $y$ ao argumento $x$:
        \[
            Nyx \equiv (\lambda sz.s(s(\ldots s(z)))yx \equiv y(y(\ldots y(x)))
        \]
    \end{itemize}

\end{frame}

\begin{frame}[fragile]{Observações sobre a função $\mathbf{Z}$}

    \begin{itemize}
        \item Assim,
        \begin{align*}
            Z0 &\equiv (\lambda x.xF\lnot F)0 \\
            &\equiv 0F\lnot F \equiv (0F\lnot)F \\
            &\equiv \lnot F \equiv T
        \end{align*}
            
        \item Para um natural $N$ qualquer,
        \begin{align*}
            ZN &\equiv (\lambda x.xF\lnot F)N \\
            &\equiv NF\lnot F \equiv (NF\lnot)F \\
            &\equiv \mathbf{I} F \equiv F,
        \end{align*}
 
        pois uma ou mais aplicações de $F$ ao argumento $\lnot$ resulta na identidade $\mathbf{I}$
    \end{itemize}

\end{frame}

\begin{frame}[fragile]{Pares}

    \begin{block}{Pares}
    No cálculo-$\lambda$, o \textbf{par} $(a, b)$ pode ser representado pela expressão-$\lambda$
    \[
        (a, b) \equiv \lambda z.zab
    \]

    O \textbf{primeiro elemento} do par pode ser extraído a partir da aplicação desta expressão ao
        termo $T$:
    \[
        (\lambda z.zab)T \equiv Tab \equiv a
    \]

    O \textbf{segundo elemento} é extraído por meio da aplicação da expressão ao termo $F$:
    \[
        (\lambda z.zab)F \equiv Fab \equiv b
    \]
    \end{block}

\end{frame}

\begin{frame}[fragile]{Par sucessor}

    \begin{block}{Proposição}
    O termo-$\lambda$
    \[
        \Phi \equiv (\lambda pz.z(S(pT))(pT))
    \]
    transforma o par $(n, n - 1)$ no par $(n + 1, n)$.
    \end{block}

    \begin{block}{Demonstração}
    De fato, o termo $(pT)$ extrai o primeiro elemento do par, e o termo $S(pT)$ é o sucessor
    deste elemento. Assim,
    \begin{align*}
        \Phi ((n, n - 1)) &\equiv  (\lambda pz.z(S(pT))(pT))(\lambda z.z(n)(n-1)) \\
        &\equiv  (\lambda z.z(S((\lambda z.z(n)(n-1))T))((\lambda z.z(n)(n-1))T)) \\
        &\equiv  (\lambda z.z(S(Tn(n-1)))(Tn(n-1))) \\
        &\equiv  (\lambda z.z(Sn)n) \equiv  (\lambda z.z(n + 1)n) \equiv (n + 1, n)
    \end{align*}
    \end{block}

\end{frame}

\begin{frame}[fragile]{Antecessor}

    \begin{block}{Proposição}
    A expressão-$\lambda$
    \[
        P \equiv (\lambda n.n\Phi (\lambda z.z00)F)
    \]
    computa o \textbf{antecessor} de qualquer natural $N$ maior que zero, e $P0 = 0$.
    \end{block}

    \begin{block}{Demonstração}
    Temos que
    \[
        P0 \equiv (\lambda n.n\Phi (\lambda z.z00)F)0 \equiv 0\Phi (\lambda z.z00) F
        \equiv (\lambda z.z00)F \equiv 0
    \]
    Seja $N$ um natural maior do que zero. Daí
    \begin{align*}
        PN &\equiv (\lambda n.n\Phi (\lambda z.z00)F)N \equiv N\Phi (\lambda z.z00) F \equiv (\lambda z.zN(N - 1))F \equiv N - 1,
    \end{align*}
    pois $N\Phi (\lambda z.z00)$ corresponde a $N$ aplicações do termo $\Phi$ ao par $(0, 0)$.
    \end{block}
\end{frame}

\begin{frame}[fragile]{Exemplo de antecessor}

    \begin{align*}
        P3 &\equiv ((\lambda n.n\Phi (\lambda z.z00))F)3 \\
           &\equiv 3\Phi (\lambda z.z00) F \\
           &\equiv (\Phi (\Phi (\Phi (\lambda z.z00)))) F \\
           &\equiv (\Phi (\Phi (\lambda z.z10))) F \\
           &\equiv (\Phi (\lambda z.z21)) F \\
           &\equiv (\lambda z.z32) F \\
           &\equiv 2
    \end{align*}

\end{frame}

\begin{frame}[fragile]{Desigualdade}

    \begin{block}{Relação maior ou igual que}
    Sejam $x$ e $y$ dois números naturais. O termo-$\lambda$
    \[
        G \equiv (\lambda xy.Z(xPy))
    \]
    retorna verdadeiro ($T$) se $x$ é \textbf{maior ou igual que} $y$, ou falso ($F$), caso 
    contrário.
    \end{block}

    \vspace{0.1in}

    \textbf{Observação}: o termo $G$ pode ser interpretado da seguinte maneira: se o resultado de
    se aplicar $x$ vezes o antecessor $P$ no natural $y$ é zero, então $x\geq y$. Este 
    procedimento remete à subtração $y - x$, exceto pelo fato de que o resultado será zero
    caso $x$ seja maior do que $y$, e não o número negativo correspondente nos inteiros.

\end{frame}

\begin{frame}[fragile]{Igualdade}

    \begin{block}{Relação igual a}
    O termo-$\lambda$
    \[
        E\equiv (\lambda xy.\land (Z(xPy))(Z(yPx)))
    \]
    retorna verdadeiro ($T$) se $x$ e $y$ são números naturais iguais, e falso ($F$), caso
    contrário.
    \end{block}

    \vspace{0.1in}

    \textbf{Observação}: $x = y$ se $x\geq y$ e $y\geq x$. Esta propriedade pode ser observada na
    definição da expressão $E$, na qual aparecem o termo-$\lambda$ correspondente à operação
    lógica \textbf{e} ($\land$) e termos oriundos da definição da desigualdade maior ou igual que
    ($G$).
\end{frame}

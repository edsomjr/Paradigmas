\section{Números Naturais}

\begin{frame}[fragile]{Contexto}

    \begin{itemize}
        \item É natural que uma linguagem de programação seja capaz de realizar operações
            aritméticas com números inteiros

        \item Contudo, conforme dito anteriormente, o cálculo $\lambda$ contém apenas dois
            termos primitivos: o símbolo $\lambda$ e o ponto final

        \item Assim, como no caso dos valores lógicos, é preciso representar os números naturais
            por meio de expressões-$\lambda$

        \item Como os inteiros são infinitos, é preciso definir uma forma de deduzir todos eles
            a partir de algum valor inicial
    \end{itemize}

\end{frame}

\begin{frame}[fragile]{Zero}

    \begin{block}{Definição de zero}
        O número natural \textbf{zero} pode ser representado pelo termo-$\lambda$
        \[
            0 \equiv \lambda sz.z
        \]
    \end{block}

    \vspace{0.1in}

    \textbf{Observação}: veja que, de acordo com a definição, acima $0 \equiv F$, onde $F$ é o
        valor lógico falso.
\end{frame}

\begin{frame}[fragile]{Sucessor}

    \begin{block}{Sucessor}
        O termo-$\lambda$
        \[
            S \equiv \lambda wyx.y(wyx)
        \]
        é denominado função sucessor, ou simplesmente, \code{cpp}{sucessor}.
    \end{block}

    \vspace{0.2in}

    \textbf{Observação}: a função sucessor permite a definição de todos os números naturais 
        a partir do zero: $1\equiv S0, 2\equiv S1, \ldots$.
\end{frame}

\begin{frame}[fragile]{Definição de $1$}

    \begin{align*}
        1 &\equiv S0 \\
          &\equiv (\lambda wyx.y(wyx))(\lambda sz.z) \\
          &\equiv (\lambda w.(\lambda yx.y(wyx)))(\lambda sz.z) \\
          &\equiv (\lambda yx.y(wyx))[w:=(\lambda sz.z)] \\
          &\equiv \lambda yx.y((\lambda sz.z)yx) \\
          &\equiv \lambda yx.y(x) \\
          &\equiv \lambda sz.s(z)
    \end{align*}

    \vspace{0.1in}

    \textbf{Observação}: no último passo foi aplicada uma conversão-$\alpha$ para renomear
        as variáveis $y$ e $x$, de modo a manter as variáveis $s$ e $z$ nas definições dos
        números naturais
\end{frame}

\begin{frame}[fragile]{Definição de $2$}

    \begin{align*}
        2 &\equiv S1 \\
          &\equiv (\lambda wyx.y(wyx))(\lambda sz.s(z)) \\
          &\equiv (\lambda w.(\lambda yx.y(wyx)))(\lambda sz.s(z)) \\
          &\equiv (\lambda yx.y(wyx))[w:=(\lambda sz.s(z))] \\
          &\equiv \lambda yx.y((\lambda sz.s(z))yx) \\
          &\equiv \lambda yx.y(y(x)) \\
          &\equiv \lambda sz.s(s(z))
    \end{align*}

    \vspace{0.1in}

    \textbf{Observação}: a definição dos naturais pode interpretada como composições de funções.
        Se $s$ é uma função, $0$ significa simplesmente retornar o argumento $z$; $1$ 
        significa aplicar a função uma vez $s(z)$; $2$ significa aplicar a função duas vezes:
        $s(s(z)) = s^2(s)$, e assim por diante. 
\end{frame}

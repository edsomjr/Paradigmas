\section{Relações entre números naturais}

\begin{frame}[fragile]{Teste Condicional}

    \begin{block}{Função $\mathbf{Z}$}
        O termo-$\lambda$
        \[
            Z\equiv \lambda x.xF\lnot F
        \]
        o qual chamaremos \textbf{função} $\mathbf{Z}$, retorna verdadeiro ($T$) quando aplicada
        em $0$, e retorna falso ($F$) para qualquer outro número natural.
    \end{block}

\end{frame}

\begin{frame}[fragile]{Observações sobre a função $\mathbf{Z}$}

    \begin{itemize}
        \item Para entender o comportamento da função $\mathbf{Z}$, observe que
        \[
            Oyx \equiv (\lambda sz.z)yx \equiv x,
        \]
        isto é, quando aplicada ao termo $xy$, $0$ ignora a ``função'' $y$ e retorna o argumento
        $x$

        \item O termo-$\lambda$ $F$, quando aplicado em qualquer termo lambda $z$, 
        retorna a identidade $\mathbf{I}$, pois
        \[
            Fz \equiv (\lambda xy.y)z \equiv (\lambda x.(\lambda y.y))z \equiv (\lambda y.y)[x:=z]
                \equiv \lambda y.y \equiv \mathbf{I}
        \]

        \item O natural $N$ aplica $N$ vezes o termo $y$ ao argumento $x$:
        \[
            Nyx \equiv (\lambda sz.s(s(\ldots s(z)))yx \equiv y(y(\ldots y(x)))
        \]
    \end{itemize}

\end{frame}

\begin{frame}[fragile]{Observações sobre a função $\mathbf{Z}$}

    \begin{itemize}
        \item Assim,
        \begin{align*}
            Z0 &\equiv (\lambda.xF\lnot F)0 \\
            &\equiv 0F\lnot F \equiv (0F\lnot)F \\
            &\equiv \lnot F \equiv T
        \end{align*}
            
        \item Para um natural $N$ qualquer,
        \begin{align*}
            ZN &\equiv (\lambda.xF\lnot F)N \\
            &\equiv NF\lnot F \equiv (NF\lnot)F \\
            &\equiv \mathbf{I} F \equiv F,
        \end{align*}
 
        pois uma ou mais aplicações de $F$ ao argumento $\lnot$ resulta na identidade $\mathbf{I}$
    \end{itemize}

\end{frame}

\documentclass{whiteboard}
\begin{document}
\begin{frame}[plain,t]
\bbcover{Programação Lógica}{Lógica Proposicional Booleana}{Prof. Edson Alves}{Campus UnB Gama: Faculdade de Ciências e Tecnologias em Engenharia}

\end{frame}
\begin{frame}[plain,t]
\begin{tikzpicture}
\node[draw,opacity=0] at (0, 0) {x};
\node[draw,opacity=0] at (14, 8) {x};

	\node[anchor=west] (title) at (0.0, 7.0) { \Large \bbbold{George Boole} };

	\node[] (moses) at (3.0, 4.0) { \includegraphics[scale=0.2]{figs/george-boole.png} };

	\node[anchor=west] (flag) at (1.2, 1.2) { \includegraphics[scale=0.005]{figs/england.png} };

	\node[] (dates) at (3.5, 1.2) { * \bbtext{1815} \hspace*{0.05in} \bbtext{\textdagger\ 1864} };

\end{tikzpicture}
\end{frame}
\begin{frame}[plain,t]
\begin{tikzpicture}
\node[draw,opacity=0] at (0, 0) {x};
\node[draw,opacity=0] at (14, 8) {x};

	\node[anchor=west] (title) at (0.0, 7.0) { \Large \bbbold{George Boole} };

	\node[] (moses) at (3.0, 4.0) { \includegraphics[scale=0.2]{figs/george-boole.png} };

	\node[anchor=west] (flag) at (1.2, 1.2) { \includegraphics[scale=0.005]{figs/england.png} };

	\node[] (dates) at (3.5, 1.2) { * \bbtext{1815} \hspace*{0.05in} \bbtext{\textdagger\ 1864} };


	\node[anchor=west] (article) at (6.0, 6.0) { \large {\bbnote{The Mathematic Analysis of Logic (1847)}} };

\end{tikzpicture}
\end{frame}
\begin{frame}[plain,t]
\begin{tikzpicture}
\node[draw,opacity=0] at (0, 0) {x};
\node[draw,opacity=0] at (14, 8) {x};

	\node[anchor=west] (title) at (0.0, 7.0) { \Large \bbbold{George Boole} };

	\node[] (moses) at (3.0, 4.0) { \includegraphics[scale=0.2]{figs/george-boole.png} };

	\node[anchor=west] (flag) at (1.2, 1.2) { \includegraphics[scale=0.005]{figs/england.png} };

	\node[] (dates) at (3.5, 1.2) { * \bbtext{1815} \hspace*{0.05in} \bbtext{\textdagger\ 1864} };


	\node[anchor=west] (article) at (6.0, 6.0) { \large {\bbnote{The Mathematic Analysis of Logic (1847)}} };


	\node[anchor=west] (a) at (6.5, 5.0) { $\star$ \bbtext{Proposta de formalização da lógica por meio da} };

	\node[anchor=west] (b) at (6.0, 4.5) { \bbtext{matemática} };

\end{tikzpicture}
\end{frame}
\begin{frame}[plain,t]
\begin{tikzpicture}
\node[draw,opacity=0] at (0, 0) {x};
\node[draw,opacity=0] at (14, 8) {x};

	\node[anchor=west] (title) at (0.0, 7.0) { \Large \bbbold{George Boole} };

	\node[] (moses) at (3.0, 4.0) { \includegraphics[scale=0.2]{figs/george-boole.png} };

	\node[anchor=west] (flag) at (1.2, 1.2) { \includegraphics[scale=0.005]{figs/england.png} };

	\node[] (dates) at (3.5, 1.2) { * \bbtext{1815} \hspace*{0.05in} \bbtext{\textdagger\ 1864} };


	\node[anchor=west] (article) at (6.0, 6.0) { \large {\bbnote{The Mathematic Analysis of Logic (1847)}} };


	\node[anchor=west] (a) at (6.5, 5.0) { $\star$ \bbtext{Proposta de formalização da lógica por meio da} };

	\node[anchor=west] (b) at (6.0, 4.5) { \bbtext{matemática} };


	\node[anchor=west] (c) at (6.5, 3.5) { $\star$ \bbtext{O livro introduz os fundamentos da lógica} };

	\node[anchor=west] (c1) at (6.0, 3.0) { \bbtext{proposicional booleana} };

\end{tikzpicture}
\end{frame}
\begin{frame}[plain,t]
\begin{tikzpicture}
\node[draw,opacity=0] at (0, 0) {x};
\node[draw,opacity=0] at (14, 8) {x};

	\node[anchor=west] (title) at (0.0, 7.0) { \Large \bbbold{George Boole} };

	\node[] (moses) at (3.0, 4.0) { \includegraphics[scale=0.2]{figs/george-boole.png} };

	\node[anchor=west] (flag) at (1.2, 1.2) { \includegraphics[scale=0.005]{figs/england.png} };

	\node[] (dates) at (3.5, 1.2) { * \bbtext{1815} \hspace*{0.05in} \bbtext{\textdagger\ 1864} };


	\node[anchor=west] (article) at (6.0, 6.0) { \large {\bbnote{The Mathematic Analysis of Logic (1847)}} };


	\node[anchor=west] (a) at (6.5, 5.0) { $\star$ \bbtext{Proposta de formalização da lógica por meio da} };

	\node[anchor=west] (b) at (6.0, 4.5) { \bbtext{matemática} };


	\node[anchor=west] (c) at (6.5, 3.5) { $\star$ \bbtext{O livro introduz os fundamentos da lógica} };

	\node[anchor=west] (c1) at (6.0, 3.0) { \bbtext{proposicional booleana} };


	\node[anchor=west] (d) at (6.5, 2.0) { $\star$ \bbtext{Ele resgata e expande estes fundamentos no seu} };

	\node[anchor=west] (d1) at (6.0, 1.5) { \bbtext{livro mais conhecido, } \bbnote{An Investigation of the Laws} };

	\node[anchor=west] (d2) at (6.0, 1.0) { \bbnote{of Thought (1849)} };

\end{tikzpicture}
\end{frame}
\begin{frame}[plain,t]
\begin{tikzpicture}
\node[draw,opacity=0] at (0, 0) {x};
\node[draw,opacity=0] at (14, 8) {x};

	\node[anchor=west] (title) at (0.0, 7.0) { \Large \bbbold{Lógica Proposicional Booleana} };

\end{tikzpicture}
\end{frame}
\begin{frame}[plain,t]
\begin{tikzpicture}
\node[draw,opacity=0] at (0, 0) {x};
\node[draw,opacity=0] at (14, 8) {x};

	\node[anchor=west] (title) at (0.0, 7.0) { \Large \bbbold{Lógica Proposicional Booleana} };


	\node[anchor=west] (terms) at (0.5, 5.0) { \bbemph{Termos primitivos} };

\end{tikzpicture}
\end{frame}
\begin{frame}[plain,t]
\begin{tikzpicture}
\node[draw,opacity=0] at (0, 0) {x};
\node[draw,opacity=0] at (14, 8) {x};

	\node[anchor=west] (title) at (0.0, 7.0) { \Large \bbbold{Lógica Proposicional Booleana} };


	\node[anchor=west] (terms) at (0.5, 5.0) { \bbemph{Termos primitivos} };


	\node[anchor=west] (a) at (1.0, 4.0) { $\star$ \bbtext{Proposição} };

\end{tikzpicture}
\end{frame}
\begin{frame}[plain,t]
\begin{tikzpicture}
\node[draw,opacity=0] at (0, 0) {x};
\node[draw,opacity=0] at (14, 8) {x};

	\node[anchor=west] (title) at (0.0, 7.0) { \Large \bbbold{Lógica Proposicional Booleana} };


	\node[anchor=west] (terms) at (0.5, 5.0) { \bbemph{Termos primitivos} };


	\node[anchor=west] (a) at (1.0, 4.0) { $\star$ \bbtext{Proposição} };


	\node[anchor=west] (b) at (1.0, 3.0) { $\star$ \bbtext{Verdadeiro} };

\end{tikzpicture}
\end{frame}
\begin{frame}[plain,t]
\begin{tikzpicture}
\node[draw,opacity=0] at (0, 0) {x};
\node[draw,opacity=0] at (14, 8) {x};

	\node[anchor=west] (title) at (0.0, 7.0) { \Large \bbbold{Lógica Proposicional Booleana} };


	\node[anchor=west] (terms) at (0.5, 5.0) { \bbemph{Termos primitivos} };


	\node[anchor=west] (a) at (1.0, 4.0) { $\star$ \bbtext{Proposição} };


	\node[anchor=west] (b) at (1.0, 3.0) { $\star$ \bbtext{Verdadeiro} };


	\node[anchor=west] (c) at (1.0, 2.0) { $\star$ \bbtext{Falso} };


\end{tikzpicture}
\end{frame}
\begin{frame}[plain,t]
\begin{tikzpicture}
\node[draw,opacity=0] at (0, 0) {x};
\node[draw,opacity=0] at (14, 8) {x};

	\node[anchor=west] (title) at (0.0, 7.0) { \Large \bbbold{Lógica Proposicional Booleana} };


	\node[anchor=west] (terms) at (0.5, 5.0) { \bbemph{Termos primitivos} };


	\node[anchor=west] (a) at (1.0, 4.0) { $\star$ \bbtext{Proposição} };


	\node[anchor=west] (b) at (1.0, 3.0) { $\star$ \bbtext{Verdadeiro} };


	\node[anchor=west] (c) at (1.0, 2.0) { $\star$ \bbtext{Falso} };



	\node[anchor=west] (axioms) at (7.0, 5.0) { \bbemph{Axiomas} };

\end{tikzpicture}
\end{frame}
\begin{frame}[plain,t]
\begin{tikzpicture}
\node[draw,opacity=0] at (0, 0) {x};
\node[draw,opacity=0] at (14, 8) {x};

	\node[anchor=west] (title) at (0.0, 7.0) { \Large \bbbold{Lógica Proposicional Booleana} };


	\node[anchor=west] (terms) at (0.5, 5.0) { \bbemph{Termos primitivos} };


	\node[anchor=west] (a) at (1.0, 4.0) { $\star$ \bbtext{Proposição} };


	\node[anchor=west] (b) at (1.0, 3.0) { $\star$ \bbtext{Verdadeiro} };


	\node[anchor=west] (c) at (1.0, 2.0) { $\star$ \bbtext{Falso} };



	\node[anchor=west] (axioms) at (7.0, 5.0) { \bbemph{Axiomas} };


	\node[anchor=west] (d) at (7.5, 4.0) { $\star$ \bbtext{Princípio do terceiro excluído} };

\end{tikzpicture}
\end{frame}
\begin{frame}[plain,t]
\begin{tikzpicture}
\node[draw,opacity=0] at (0, 0) {x};
\node[draw,opacity=0] at (14, 8) {x};

	\node[anchor=west] (title) at (0.0, 7.0) { \Large \bbbold{Lógica Proposicional Booleana} };


	\node[anchor=west] (terms) at (0.5, 5.0) { \bbemph{Termos primitivos} };


	\node[anchor=west] (a) at (1.0, 4.0) { $\star$ \bbtext{Proposição} };


	\node[anchor=west] (b) at (1.0, 3.0) { $\star$ \bbtext{Verdadeiro} };


	\node[anchor=west] (c) at (1.0, 2.0) { $\star$ \bbtext{Falso} };



	\node[anchor=west] (axioms) at (7.0, 5.0) { \bbemph{Axiomas} };


	\node[anchor=west] (d) at (7.5, 4.0) { $\star$ \bbtext{Princípio do terceiro excluído} };


	\node[anchor=west] (e) at (7.5, 3.0) { $\star$ \bbtext{Princípio da não-contradição} };

\end{tikzpicture}
\end{frame}
\begin{frame}[plain,t]
\begin{tikzpicture}
\node[draw,opacity=0] at (0, 0) {x};
\node[draw,opacity=0] at (14, 8) {x};

	\node[anchor=west] (title) at (0.0, 7.0) { \Large \bbbold{Prolog (1972)} };

\end{tikzpicture}
\end{frame}
\begin{frame}[plain,t]
\begin{tikzpicture}
\node[draw,opacity=0] at (0, 0) {x};
\node[draw,opacity=0] at (14, 8) {x};

	\node[anchor=west] (title) at (0.0, 7.0) { \Large \bbbold{Prolog (1972)} };


	\node[] (a) at (4.0, 6.0) { \bbemph{Proponentes} };

\end{tikzpicture}
\end{frame}
\begin{frame}[plain,t]
\begin{tikzpicture}
\node[draw,opacity=0] at (0, 0) {x};
\node[draw,opacity=0] at (14, 8) {x};

	\node[anchor=west] (title) at (0.0, 7.0) { \Large \bbbold{Prolog (1972)} };


	\node[] (a) at (4.0, 6.0) { \bbemph{Proponentes} };


	\node[] (alain) at (2.0, 3.0) { \includegraphics[scale=0.35]{figs/alain.jpg} };

	\node[] (alain_name) at (2.0, 0.5) { \bbtext{Alain Colmerauer} };

\end{tikzpicture}
\end{frame}
\begin{frame}[plain,t]
\begin{tikzpicture}
\node[draw,opacity=0] at (0, 0) {x};
\node[draw,opacity=0] at (14, 8) {x};

	\node[anchor=west] (title) at (0.0, 7.0) { \Large \bbbold{Prolog (1972)} };


	\node[] (a) at (4.0, 6.0) { \bbemph{Proponentes} };


	\node[] (alain) at (2.0, 3.0) { \includegraphics[scale=0.35]{figs/alain.jpg} };

	\node[] (alain_name) at (2.0, 0.5) { \bbtext{Alain Colmerauer} };


	\node[] (philippe) at (6.5, 3.0) { \includegraphics[scale=0.4]{figs/philippe.jpg} };

	\node[] (philippe_name) at (6.5, 0.5) { \bbtext{Philippe Roussel} };

\end{tikzpicture}
\end{frame}
\begin{frame}[plain,t]
\begin{tikzpicture}
\node[draw,opacity=0] at (0, 0) {x};
\node[draw,opacity=0] at (14, 8) {x};

	\node[anchor=west] (title) at (0.0, 7.0) { \Large \bbbold{Prolog (1972)} };


	\node[] (a) at (4.0, 6.0) { \bbemph{Proponentes} };


	\node[] (alain) at (2.0, 3.0) { \includegraphics[scale=0.35]{figs/alain.jpg} };

	\node[] (alain_name) at (2.0, 0.5) { \bbtext{Alain Colmerauer} };


	\node[] (philippe) at (6.5, 3.0) { \includegraphics[scale=0.4]{figs/philippe.jpg} };

	\node[] (philippe_name) at (6.5, 0.5) { \bbtext{Philippe Roussel} };


	\node[] (b) at (11.0, 6.0) { \bbemph{Inspiração} };

\end{tikzpicture}
\end{frame}
\begin{frame}[plain,t]
\begin{tikzpicture}
\node[draw,opacity=0] at (0, 0) {x};
\node[draw,opacity=0] at (14, 8) {x};

	\node[anchor=west] (title) at (0.0, 7.0) { \Large \bbbold{Prolog (1972)} };


	\node[] (a) at (4.0, 6.0) { \bbemph{Proponentes} };


	\node[] (alain) at (2.0, 3.0) { \includegraphics[scale=0.35]{figs/alain.jpg} };

	\node[] (alain_name) at (2.0, 0.5) { \bbtext{Alain Colmerauer} };


	\node[] (philippe) at (6.5, 3.0) { \includegraphics[scale=0.4]{figs/philippe.jpg} };

	\node[] (philippe_name) at (6.5, 0.5) { \bbtext{Philippe Roussel} };


	\node[] (b) at (11.0, 6.0) { \bbemph{Inspiração} };


	\node[] (robert) at (11.0, 3.0) { \includegraphics[scale=0.35]{figs/robert.jpg} };

	\node[] (robert_name) at (11.0, 0.5) { \bbtext{Robert Kowalski} };

\end{tikzpicture}
\end{frame}
\begin{frame}[plain,t]
\begin{tikzpicture}
\node[draw,opacity=0] at (0, 0) {x};
\node[draw,opacity=0] at (14, 8) {x};

	\node[anchor=west] (title) at (0.0, 7.0) { \Large \bbbold{SWI Prolog} };

\end{tikzpicture}
\end{frame}
\begin{frame}[plain,t]
\begin{tikzpicture}
\node[draw,opacity=0] at (0, 0) {x};
\node[draw,opacity=0] at (14, 8) {x};

	\node[anchor=west] (title) at (0.0, 7.0) { \Large \bbbold{SWI Prolog} };


	\node[anchor=west] (a) at (1.0, 6.0) { $\star$ \bbtext{ Prolog é uma contração da expressão ``PROgramming in LOGic''} };

\end{tikzpicture}
\end{frame}
\begin{frame}[plain,t]
\begin{tikzpicture}
\node[draw,opacity=0] at (0, 0) {x};
\node[draw,opacity=0] at (14, 8) {x};

	\node[anchor=west] (title) at (0.0, 7.0) { \Large \bbbold{SWI Prolog} };


	\node[anchor=west] (a) at (1.0, 6.0) { $\star$ \bbtext{ Prolog é uma contração da expressão ``PROgramming in LOGic''} };


	\node[anchor=west] (b) at (1.0, 5.0) { $\star$ \bbtext{Tem raízes na lógica de primeira ordem} };

\end{tikzpicture}
\end{frame}
\begin{frame}[plain,t]
\begin{tikzpicture}
\node[draw,opacity=0] at (0, 0) {x};
\node[draw,opacity=0] at (14, 8) {x};

	\node[anchor=west] (title) at (0.0, 7.0) { \Large \bbbold{SWI Prolog} };


	\node[anchor=west] (a) at (1.0, 6.0) { $\star$ \bbtext{ Prolog é uma contração da expressão ``PROgramming in LOGic''} };


	\node[anchor=west] (b) at (1.0, 5.0) { $\star$ \bbtext{Tem raízes na lógica de primeira ordem} };


	\node[anchor=west] (c1) at (1.0, 4.0) { $\star$ \bbtext{O SWI-Prolog pode ser instalado por meio do comando} };

	\node[anchor=west] (c2) at (2.0, 3.0) { \mintinline{bash}{$ sudo apt-get install swi-prolog} };

\end{tikzpicture}
\end{frame}
\begin{frame}[plain,t]
\begin{tikzpicture}
\node[draw,opacity=0] at (0, 0) {x};
\node[draw,opacity=0] at (14, 8) {x};

	\node[anchor=west] (title) at (0.0, 7.0) { \Large \bbbold{SWI Prolog} };


	\node[anchor=west] (a) at (1.0, 6.0) { $\star$ \bbtext{ Prolog é uma contração da expressão ``PROgramming in LOGic''} };


	\node[anchor=west] (b) at (1.0, 5.0) { $\star$ \bbtext{Tem raízes na lógica de primeira ordem} };


	\node[anchor=west] (c1) at (1.0, 4.0) { $\star$ \bbtext{O SWI-Prolog pode ser instalado por meio do comando} };

	\node[anchor=west] (c2) at (2.0, 3.0) { \mintinline{bash}{$ sudo apt-get install swi-prolog} };


	\node[anchor=west] (d1) at (1.0, 2.0) { $\star$ \bbtext{O interpretador (}\bbenglish{listener}\bbtext{) Prolog pode ser invocado com o comando} };

	\node[anchor=west] (d2) at (2.0, 1.0) { \mintinline{bash}{$ prolog} };

\end{tikzpicture}
\end{frame}
\begin{frame}[plain,t]
\begin{tikzpicture}
\node[draw,opacity=0] at (0, 0) {x};
\node[draw,opacity=0] at (14, 8) {x};

	\node[anchor=west] (title) at (0.0, 7.0) { \Large \bbbold{Valores lógicos em Prolog} };

\end{tikzpicture}
\end{frame}
\begin{frame}[plain,t]
\begin{tikzpicture}
\node[draw,opacity=0] at (0, 0) {x};
\node[draw,opacity=0] at (14, 8) {x};

	\node[anchor=west] (title) at (0.0, 7.0) { \Large \bbbold{Valores lógicos em Prolog} };


	\node[anchor=west] (a1) at (1.0, 6.0) { $\star$ \bbtext{Prolog implementa os termos primitivos} \bbenglish{verdadeiro} \bbtext{e} \bbenglish{falso} \bbtext{por meio dos} };

	\node[anchor=west] (a2) at (0.5, 5.5) { \bbtext{predicados \code{prolog}{true/0} e \code{prolog}{false/0}} };

\end{tikzpicture}
\end{frame}
\begin{frame}[plain,t]
\begin{tikzpicture}
\node[draw,opacity=0] at (0, 0) {x};
\node[draw,opacity=0] at (14, 8) {x};

	\node[anchor=west] (title) at (0.0, 7.0) { \Large \bbbold{Valores lógicos em Prolog} };


	\node[anchor=west] (a1) at (1.0, 6.0) { $\star$ \bbtext{Prolog implementa os termos primitivos} \bbenglish{verdadeiro} \bbtext{e} \bbenglish{falso} \bbtext{por meio dos} };

	\node[anchor=west] (a2) at (0.5, 5.5) { \bbtext{predicados \code{prolog}{true/0} e \code{prolog}{false/0}} };


	\node[anchor=west] (b) at (2.0, 4.0) { \inputsyntax{prolog}{codes/true_false.pl} };

\end{tikzpicture}
\end{frame}
\begin{frame}[plain,t]
\begin{tikzpicture}
\node[draw,opacity=0] at (0, 0) {x};
\node[draw,opacity=0] at (14, 8) {x};

	\node[anchor=west] (title) at (0.0, 7.0) { \Large \bbbold{Valores lógicos em Prolog} };


	\node[anchor=west] (a1) at (1.0, 6.0) { $\star$ \bbtext{Prolog implementa os termos primitivos} \bbenglish{verdadeiro} \bbtext{e} \bbenglish{falso} \bbtext{por meio dos} };

	\node[anchor=west] (a2) at (0.5, 5.5) { \bbtext{predicados \code{prolog}{true/0} e \code{prolog}{false/0}} };


	\node[anchor=west] (b) at (2.0, 4.0) { \inputsyntax{prolog}{codes/true_false.pl} };


	\node[anchor=west] (c) at (1.0, 2.5) { $\star$ \bbtext{Prolog faz distinção entre maiúsculas e minúsculas} };


\end{tikzpicture}
\end{frame}
\begin{frame}[plain,t]
\begin{tikzpicture}
\node[draw,opacity=0] at (0, 0) {x};
\node[draw,opacity=0] at (14, 8) {x};

	\node[anchor=west] (title) at (0.0, 7.0) { \Large \bbbold{Valores lógicos em Prolog} };


	\node[anchor=west] (a1) at (1.0, 6.0) { $\star$ \bbtext{Prolog implementa os termos primitivos} \bbenglish{verdadeiro} \bbtext{e} \bbenglish{falso} \bbtext{por meio dos} };

	\node[anchor=west] (a2) at (0.5, 5.5) { \bbtext{predicados \code{prolog}{true/0} e \code{prolog}{false/0}} };


	\node[anchor=west] (b) at (2.0, 4.0) { \inputsyntax{prolog}{codes/true_false.pl} };


	\node[anchor=west] (c) at (1.0, 2.5) { $\star$ \bbtext{Prolog faz distinção entre maiúsculas e minúsculas} };



	\node[anchor=west] (d) at (2.0, 1.25) { \inputsyntax{prolog}{codes/42.pl} };


\end{tikzpicture}
\end{frame}
\begin{frame}[plain,t]
\begin{tikzpicture}
\node[draw,opacity=0] at (0, 0) {x};
\node[draw,opacity=0] at (14, 8) {x};

	\node[anchor=west] (title) at (0.0, 6.5) { \Large \bbbold{Referências} };

	\node[anchor=west] (a) at (1.0, 5.0) { $\star$ \bbbold{SWI-Prolog.} \bbenglish{https://www.swi-prolog.org/,} \bbtext{acesso em 10/02/2026.} };

	\node[anchor=west] (b) at (1.0, 4.0) { $\star$ \bbbold{WOLFRAM}\bbtext{, Stephen.} \bbenglish{George Boole: A 200-Year View,} \bbtext{acesso em 10/02/2026.} };


\end{tikzpicture}
\end{frame}
\end{document}
